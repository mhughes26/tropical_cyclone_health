% Options for packages loaded elsewhere
\PassOptionsToPackage{unicode}{hyperref}
\PassOptionsToPackage{hyphens}{url}
%
\documentclass[
]{article}
\usepackage{lmodern}
\usepackage{amssymb,amsmath}
\usepackage{ifxetex,ifluatex}
\ifnum 0\ifxetex 1\fi\ifluatex 1\fi=0 % if pdftex
  \usepackage[T1]{fontenc}
  \usepackage[utf8]{inputenc}
  \usepackage{textcomp} % provide euro and other symbols
\else % if luatex or xetex
  \usepackage{unicode-math}
  \defaultfontfeatures{Scale=MatchLowercase}
  \defaultfontfeatures[\rmfamily]{Ligatures=TeX,Scale=1}
\fi
% Use upquote if available, for straight quotes in verbatim environments
\IfFileExists{upquote.sty}{\usepackage{upquote}}{}
\IfFileExists{microtype.sty}{% use microtype if available
  \usepackage[]{microtype}
  \UseMicrotypeSet[protrusion]{basicmath} % disable protrusion for tt fonts
}{}
\makeatletter
\@ifundefined{KOMAClassName}{% if non-KOMA class
  \IfFileExists{parskip.sty}{%
    \usepackage{parskip}
  }{% else
    \setlength{\parindent}{0pt}
    \setlength{\parskip}{6pt plus 2pt minus 1pt}}
}{% if KOMA class
  \KOMAoptions{parskip=half}}
\makeatother
\usepackage{xcolor}
\IfFileExists{xurl.sty}{\usepackage{xurl}}{} % add URL line breaks if available
\IfFileExists{bookmark.sty}{\usepackage{bookmark}}{\usepackage{hyperref}}
\hypersetup{
  pdftitle={Challenges of Integrating Physical Exposure and Human Impacts Data in Tropical Cyclone Studies},
  pdfauthor={Matthew Hughes and Brooke Anderson},
  hidelinks,
  pdfcreator={LaTeX via pandoc}}
\urlstyle{same} % disable monospaced font for URLs
\usepackage[margin=1in]{geometry}
\usepackage{graphicx,grffile}
\makeatletter
\def\maxwidth{\ifdim\Gin@nat@width>\linewidth\linewidth\else\Gin@nat@width\fi}
\def\maxheight{\ifdim\Gin@nat@height>\textheight\textheight\else\Gin@nat@height\fi}
\makeatother
% Scale images if necessary, so that they will not overflow the page
% margins by default, and it is still possible to overwrite the defaults
% using explicit options in \includegraphics[width, height, ...]{}
\setkeys{Gin}{width=\maxwidth,height=\maxheight,keepaspectratio}
% Set default figure placement to htbp
\makeatletter
\def\fps@figure{htbp}
\makeatother
\setlength{\emergencystretch}{3em} % prevent overfull lines
\providecommand{\tightlist}{%
  \setlength{\itemsep}{0pt}\setlength{\parskip}{0pt}}
\setcounter{secnumdepth}{-\maxdimen} % remove section numbering

\title{Challenges of Integrating Physical Exposure and Human Impacts Data in
Tropical Cyclone Studies}
\author{Matthew Hughes and Brooke Anderson}
\date{May 25, 2020}

\begin{document}
\maketitle

\hypertarget{introduction}{%
\section{Introduction}\label{introduction}}

Tropical cyclones---which encompasses hurricanes as well as tropical
storms and tropical depressions--- regularly threaten coastal
communities across the Eastern and Southern United States. From 2000 to
2019, tropical cyclones cost the United States at least 811 billion
dollars in damages({\textbf{???}} billion dollar disasters). Tropical
cyclones in that same time frame resulted in 6,010 human fatalities,
averaging 301 deaths per year ({\textbf{???}} billion dollar disasters).
Tropical cyclones upset coastal communities and society by damaging
property, disrupting local economies, and harming human health. This is
why they are so critical to study.

Tropical cyclones are environmental disaster events that are crucial for
public health authorities and scientists to understand. Human mortality
is an obvious consequence of these storms, and in 1992 Hurricane Andrew
left 53 residents in Florida and surrounding states dead (Ahrens 2005).
However, many other chronic and long term health impacts have been
observed in the aftermath of tropical cyclones. Researchers have
observed that in utero exposure to tropical cyclones leads to adverse
birth outcomes. (Kinney et al. 2008) observed higher rates of autism in
children born to mothers who had higher rates of storm exposure than
children born to mothers who were exposed to later intensities. The
scientific literature also reveals evidence of mental health outcomes
associated with populations exposed to tropical cyclones. Survivors of
tropical storms often report higher levels of depression, anxiety, and
PTSD, due to reduced access to important medical and social services,
property damages, poor sanitation, and displacement after storms.
(Lieberman-Cribbin et al. 2017) found higher levels of PTSD in New York
City residents who were exposed to flooding after Hurricane Sandy.
Beyond health impacts, both mental and physical, tropical storms create
incredible strains on the economies of the Southeastern United States.
The average cost of a tropical cyclone event in the US is 21.2 billion
per event, CPI-adjusted ({\textbf{???}} billion dollar disasters).

Clearly, tropical cyclones dramatically impact the social, economic, and
physical wellbeing of coastal communities. These extreme weather events
represent an environmental health threat that is not going to disappear,
and given that coastal regions of the Southeastern US are experiencing
population growth, it is likely that higher numbers of people will be
put at risk in the future. Avoiding these risks is not possible, but
building resilience in communities after they experience tropical
cyclone events is key to mitigating damages and preparing for future
disasters. Creating lasting and resilient communities in areas prone to
tropical cyclones requires that researchers understand which populations
and locations are at the greatest risk for negative exposures to
tropical storms. This requires data that allows researchers to assess
where in space and time tropical storms occur, and also where in space
and time individuals and populations are experiencing impacts from these
storms.

Multidisciplinary teams of researchers are exploring this using
different datasets, however a key challenge is integrating data from
across disciplines. For example: extensive physical exposure data is
often available for tropical cyclones as they near and cross communities
in the United States. This data can come both from established
monitoring networks, like {[}NOAA network name?{]}, but may also result
from data collection efforts during or after the storm by atmospheric
scientists and engineers seeking to characterize a storm's physical
properties. Researchers studying the human impacts of these storms,
including epidemiologists, economists, and social scientists are
interested in this data as well, but the differences in temporal and
spatial resolution makes the data harder to use. Resolving physical
exposure and human impact datasets is challenging because the human
impact data and physical exposure data often do not have congruent
resolutions.

Here we explore cases and implications of integrating data at different
temporal and spatial scales, focusing as an example on human impact
studies of tropical cyclones in the US. We begin by investigating the
reasons that spatial and temporal misaligment exist in the study of
tropical cyclones. We then describe the main spatial and temporal scales
used, and finally assess some of the consequences that result from
integrating physical exposure data with human impacts data.

\hypertarget{physical-exposures}{%
\subsection{Physical Exposures}\label{physical-exposures}}

Atmospheric and weather data have long been designed to give a picture
of meteorological activity over vast geographic spreads as large as
entire continents or oceanic basins. To acheive this, data is often
recorded by sensors at fixed weather monitoring stations, in vast
monitoring systems that are designed to automatically record a data
point at a fixed interval of time. These monitoring systems are often
the result of long-standing weather projects such as the National
Hurricane Center Data Archive from NOAA (National Oceanic and
Atmospheric Administration), and the NWS (National Weather Service).
This data is often narrow in temporal and spatial resolution, and large
in geographic scope.

\hypertarget{storm-tracks}{%
\subsubsection{Storm Tracks}\label{storm-tracks}}

Tropical cyclone storm tracks refer to the paths the storms take, and
can be displayed on maps to visualize where the center of the storm (the
eye of the tropical cyclone) passes through. In the North Atlantic
Basin, tropical cyclone paths have a tendency to move westward first,
then curve north and sometimes northeastward before ending, although
some storm tracks take very messy and circuitous paths. Satellite
imagery and remote sensing can be used to detect the paths of tropical
cyclones and hurricanes, and ground monitors measuring wind speed can
also detect the movement of the storm. The location of the center of a
tropical cyclone can be documented at specific point locations at
different times, meaning that it has a very narrow spatial and also
temporal resolution.

One source of data on tropical cyclone storm tracks is the Hurricane
Data second generation dataset (HURDAT2) which has information on the
point location of storm centers, wind speed, and atmospheric pressure in
six hour intervals of North Atlantic tropical cyclones since 1851
(Deryugina 2017).

Oftentimes storm tracks are used in tropical cyclone studies to assign
exposures. Sometimes, distance from a storm track is used to assign
counties or zip codes as exposed, as in ({\textbf{???}}). If distance
from storm track is used to assign exposure, a threshold will have to be
chosen to determine whether or not a county or zip code is exposed.
Larger distance thresholds will increase the number of counties defined
as exposed, and potentially overestimate exposure. Smaller distance
thresholds will decrease the number of counties defined as exposed and
potentially underestimate exposure. In either case, misclassification of
exposure could arise.

Other times, exposure to a tropical cyclone is assigned only if the
storm track passed through a county or zip code. This is the approach
that was used in (Kinney et al. 2008). Populations are not typically
distributed in a uniform pattern across these spatial areas, and so
using this method to assign exposure could also result in exposure
misclassification for communities that are close to the storm track but
not in the county or zip code that the storm track passed through. The
reverse of this, where communities in exposed counties or zip codes are
actually located farther away from the storm track is also plausible.

\hypertarget{wind-speed-and-direction}{%
\subsubsection{Wind Speed and
Direction}\label{wind-speed-and-direction}}

Wind speed is a common way to characterize exposure to tropical
cyclones. To even be classified as a tropical cyclone, a storm must have
wind speeds in excess of 74 miles per hour (64 knots). Meteorologists
and atmospheric scientists use ground based wind instruments in set
locations to measure wind speed and direction, such as wind vanes,
anenometers, and aerovanes. In order to be accurate and effective, these
ground based wind instruments must be placed above the roofs of
buildings so that they can be exposed to free flowing air. Since this is
not always the case, wind observations can consequently be erratic in
nature. Above ground, geostationary satellites, which are positioned
above a particular location can measure wind speed and wind direction by
observing the direction that clouds move in a given amount of time.
Doppler radar can also be used to measure wind speed and direction.

Wind speed is often used as a measure for assigning exposure, often by
choosing a threshold wind speed that if attained in a zip code or
county, makes it exposed. (Yan et al. 2020) used the sustained maximum
wind speed recorded at the center of counties. In this example counties
were considered exposed to tropical cyclones if the sustained maximum
wind speed was greater than 21 meters per second. Because this wind
speed is taken from a monitor at the center of the county, it may not be
representative of wind speeds in other parts of the exposed county,
again contributing to potential misclassification of exposure.

Another study that used wind speed to assign exposure was (Parks et al.
2021), which categorized counties as exposed to tropical cyclones on
days that they experienced peak sustained wind greater than or equal to
34 knots when the the cyclone was at the point of closest approach to
the county.

\hypertarget{flooding-and-storm-surges}{%
\subsection{Flooding and Storm Surges}\label{flooding-and-storm-surges}}

Due to intensive precipitation, and the threat of storm surges during
and after tropical cyclones make landfall, flooding is a major
consequence that has a number of impacts on the infrastructure, safety,
and economic strength of communities. Flooding accounts for about 75\%
of declared federal disasters, costs an average of \$8 billion in the US
annually, and results in over 90 fatalities on average each year.
{[}USGS, 2016{]}

There are several methods for measuring flooding and creating geospatial
maps to show the extent and impact of flooding. One method is to measure
high water marks. Typically, this involves sending people out to
specific locations to record the high water marks, but this method is
costly, requires intensive labor, and is difficult to acheive during or
after flooding disasters (Li et al. 2018).

Another method is to use data from stream gauges. The United States
Geological Survey (USGS) maintains stream gauges at monitored locations
along bodies of water that regularly record information on water height
and stream flow, often updating every fifteen minutes (Li et al. 2018).
There are some limitations to this method as well, for example these
stream gauges are not systematically installed along water ways, meaning
that information is not uniform, and the stream gauges are not useful if
the water level rises above the limit of ground based gauges or washes
gauges away entirely (Li et al. 2018).

Satellite imagery, aerial photography, and remote sensing can also be
used to asses the extent and damage of flooding in the aftermath of
tropical cyclone disasters, but issues pertaining to cloud cover and
inclement weather can make high quality, consistent, and clear images
difficult to acheive and therefore use in analyzing impacts (Li et al.
2018).

\hypertarget{human-impacts-of-tropical-cyclones}{%
\section{Human Impacts of Tropical
Cyclones}\label{human-impacts-of-tropical-cyclones}}

Where physical exposure data is often expansive and specific, owing to
well established networks of weather monitoring stations, data on human
impacts are spatially and temporally located within geopolitical,
cultural, and administrative boundaries. This type of data is available
often in the form of census records, hospitalization records and vitals
statistics from hospitals and public health departments, disaster
insurance claims, schools, and other systems that record human
activities. Unlike the physical exposure data, these sources are often
aggregated by geographic region and time, often out of convenience, or a
need to preserve the anonymity and privacy of the people whose data is
being used. Researchers also will use such secondary datasets and
sources to compare with primary data sources. For example in
(Lieberman-Cribbin et al. 2017), self reported flooding exposure data
was compared to FEMA flooding exposure data.

\hypertarget{health-impacts}{%
\subsection{Health Impacts}\label{health-impacts}}

Tropical cyclones studies have documented associations between exposure
to tropical cyclones and a number of health outcomes such as increased
hospitalizations due to cardiovascular and respiratory effects (Yan et
al. 2020), autism in children from in utero exposure (Kinney et al.
2008), risk of preterm birth (S. C. Grabich et al. 2016), adverse mental
health outcomes such as anxiety, depression, and PTSD (Lieberman-Cribbin
et al. 2017),(Scaramutti et al. 2019),(Bevilacqua et al. 2020),
increased risk of hypertension (Ferdinand 2005), and injury and death
(Lane et al. 2013). The health data that informed thesd studies comes
from a number of public health agencies, both governmental and
non-governmental that collect extensive information pertaining to
deaths, acute and chronic illnesses, injuries, birth and pregnancy
outcomes, and mental health conditions. This data provides researchers
with a wealth of information on health related human impacts of tropical
cyclones.

Certain general health information can be accessed from data published
by the National Vital Statistics System of the National Center for
Health Statistics (NCHS) (Aschengrau and Seage 2013). This organization
has registration offices in every U.S. state, Washington D.C., and New
York City. Vital statistics from birth certificates for example, are
recorded and verified by medical professionals and submitted to local
health departments, which submit this information to state health
departments, which eventually send it to the NCHS (Aschengrau and Seage
2013). Because local health departments typically exist at the county
and state level, this health information will also be aggregated at
those levels.

Mortality data is also collected in the US by the NCHS through a program
it administers called the National Death Index (Aschengrau and Seage
2013); this particular data has to be obtained through offices at the
state level. Death certificates themselves will give the information of
the events that led to death, something of interest when determining
impacts of tropical cyclones. There are many other sources of health
data that contain information pertinent to impacts from tropical
cyclones such as the National Health Interview Survey, National
Notifiable Diseases Surveillance System, Planned Parenthood Federation
of America, Center for Disease Control, Pregnancy Risk Assessment
Monitoring System, and many others (Aschengrau and Seage 2013). The key
is to understand that these data come from hospitals, public health
departments and other agencies at county and state levels.

\hypertarget{social-and-economic-impacts}{%
\subsection{Social and Economic
Impacts}\label{social-and-economic-impacts}}

There a wide variety of social and economic costs associated with
tropical cyclones. Often large populations of people are displaced after
tropical cyclone events, such as the Puerto Ricans who migrated to
Florida after Hurricane Maria (Scaramutti et al. 2019). Another crucial
consequence of tropical cyclones is that homes, businesses, and
commmunity infrastructure are damaged, often severly. This destruction
alters local economies, sometimes leading to unexpected economic
consequences.

To study social changes after tropical cyclones, demographic details
such as race, ethnicity, socioeconomic status, age, and political
affiliation are interesting and often insightful details of information
that can help to shine a light on the human impacts of tropical
cyclones. For many researchers, this data can be gleaned from the US
Census. The US Census is a valuable source of information that is
updated and compiled every ten years by the US Bureau of the Census on
many variables including ancestry, racial background, mortage,
occupation, household size, etc.(Aschengrau and Seage 2013).

To quantify economic impacts of tropical cyclones, there are a variety
of metrics that are used by researchers, such as studying how employment
and earnings change before and after a tropical cyclone event. (Belasen
and Polachek 2008) built a generalized difference-in-difference (GDD)
model to study the effects of hurricanes on county-level employment and
county-level average quarterly earnings per worker in the state of
Florida. Though the state of Florida was studied here, results from
looking at the county level showed differences in economic impacts
depending on the severity of the hurricane, and the intensity of it when
the county was hit by it.

Individual tax returns are another resource that researchers can use to
estimate economic impacts from tropical cyclones. Tax returns and tax
records can provide a wealth of information on the financial and
economic situations of large numbers of individuals before and after a
tropical cyclone event because they can be linked to individual
residential addresses (a point location), which allows researchers to
identify residents of an area before a tropical cyclone, and they can
give information about wages, salaries, self-employment, unemployment
insurance, the Social Security Disability Insurance program, and
retirement accounts (Deryugina, Kawano, and Levitt 2018). (Deryugina,
Kawano, and Levitt 2018) did just this to study the economic impact of
Hurricane Katrina on the city of New Orleans, by collected information
on individual federal tax returns and third party information returns
filed between 1999 and 2013. Because tax returns are linked to
individuals with known residential addresses, tax returns allow
researchers to observe economic impacts at a point location.

Another great resource researchers can utilize for studying economic
impacts of tropical cyclones is from insurance claims.

\hypertarget{spatial-and-temporal-scales-and-misalignment}{%
\section{Spatial and Temporal Scales and
Misalignment}\label{spatial-and-temporal-scales-and-misalignment}}

Questions about the human impacts of tropical cyclones are
multidisciplinary, and as such require datasets from different and
sometimes seemingly disparate sources. The physical exposures of
tropical cyclones that were mentioned above such as wind speed, storm
tracks, precipitation, and flooding data come from monitors that are at
fixed locations or at locations created by models in gridded formations.
In contrast, the data on human health, social, and economic impacts will
come from hospitals, schools, census reports, insurance claims, tax
returns, and other documents and records coming from typically more
aggregated spatial levels like counties.

Differences in spatial and temporal scales are also related to the study
question that researchers are asking. If a study is concerned with birth
outcomes for example, having weather data on the windspeed every several
seconds may not be relevant, because birth outcomes related to storm
exposure in utero may operate on a longer time scale. In (S. C. Grabich
et al. 2016), the researchers looked at gestational periods and defined
pregnancies as exposed to tropical cyclones if they happened before 20
weeks of gestation. If the researchers had been interested in a
different question, for example acute injuries due to direct storm
exposure, they would have chosen a smaller time scale. There is no
correct spatial or temporal scale that works well for all research, it
all depends on what is being asked and how that can be ascertained.
Different scales allow the researchers to make certain inferences and
determine how the results of a study can be interpreted.

The remainder of this section will highlight the most common spatial and
temporal scales typically used in tropical cyclone studies. These scales
were chosen after conducting a literature review that covered a wide
range of human impacts from tropical cyclones. First we will describe
spatial scales starting from the smallest resolution of point locations,
working up to the level of metropolitan areas and states. Next we will
describe temporal scales most commonly used in tropical cyclone studies
and again work from smallest to largest resolution.

\hypertarget{spatial-scales}{%
\section{Spatial Scales}\label{spatial-scales}}

When we refer to spatial scales, we are referring to the size of a
geographic space that data are collected from. In tropical cyclone
studies, the size of this space can be as small as a latitude-longitude
coordinate (a point location), or as large as an oceanic basin; for
example the North Atlantic Basin where North American tropical cyclones
typically develop. In general physical exposure data comes from monitors
and sensors at point locations, while human impacts data comes from
larger aggregated scales such as zip codes, counties, and states.

It can be helpful to think of geospatial data in two major ways, as
vectors, or as rasters. Vector data use points, lines, or polygons to
represent geographic features and locations (Lovelace, Nowosad, and
Muenchow 2019). These vector data are discrete and well-defined, and are
typically used to study human impacts, because administrative boundaries
and borders utilize this data type (Lovelace, Nowosad, and Muenchow
2019). For example, polygons can be used to represent closed areas such
as zip codes, counties, states, countries, islands, or even continents.
Tropical cyclone storm tracks can be thought of as as line vectors.
Vector points can represent smaller areas such as cities, mountain
peaks, locations of hospitals, etc.

The other major spatial data type is a raster. Rasters are gridded data,
meaning that they are displayed as cells that divide a surface into
equal, regularly spaced parts (Lovelace, Nowosad, and Muenchow 2019).
Raster data is often utilized for physical exposures of tropical
cyclones because it can display continuous data (such as wind speed or
flooding over a geographic area). However, because raster data can also
display categorical data, it can also be used for studying human impacts
(for example rasterized data could overlay socioeconomic makeup of a
city on a map, with each raster representing the dominant group at that
location) (Lovelace, Nowosad, and Muenchow 2019).

Here we will describe several spatial scales that are commonly used by
researchers studying human impacts of tropical cyclones in ascending
order of magnitude. With each scale we will describe how measurements
and data are generated from that particular area. We will also describe
the methods that researchers use to join physical exposures data with
human impacts data to accurately ascertain responses to exposure.

\hypertarget{vector-data}{%
\subsection{Vector Data}\label{vector-data}}

In tropical cyclone studies, the vector classes most often utilized are
self standing point locations, or polygons -- which can be thought of as
a more complex geometric collection of points.

\hypertarget{point-location}{%
\subsubsection{Point Location}\label{point-location}}

Point locations are the smallest resolution of spatial data used to
assess the exposure to tropical storms and hurricanes. Point locations
can be characterized by specific latitude and longitude values, which
specify a specific geographic location on a map. Meteorological
instruments, monitors, and sensors that collect information on physical
exposures are at this spatial level, often located at airports, weather
stations, and even personal monitors used by volunteers (example
CoCoRAHs).

When Hurricane Ike struck the coasts of Texas and Louisiana in 2008, the
U.S. Geological Survey set up 117 pressure transducers (a type of
sensor) as a temporary monitoring network spanning over 5,000 square
miles along the Gulf Coast of the affected states(East, Turco, and Mason
Jr 2008). This temporary monitoring network was designed to record the
timing, areal extent, and magnitude of inland hurricane storm surge and
coastal flooding.Although the combined network of sensors and the data
they record was used to evaluate storm surge models and document the
extent of flooding and other site specific effects, each individual
pressure transducer represented a specific point location (East, Turco,
and Mason Jr 2008).

Storm tracks, which are mapped as lines, are really composed of a series
of point locations on a map. These point locations come from known
locations based on satellite data or monitors on the ground measuring
wind speed, and then the storm track itself is interpolated to visualize
the entirety of the track. (Yan et al. 2020) used distance from storm
track to assign exposure to the tropical cyclone, and part of the
cyclone's storm track that was closest represented an individual point
location. Exposure to tropical cyclones using storm tracks was assessed
differently in (Kinney et al. 2008), the storm track simply had to pass
through a particular county for residents to be considered exposed, but
the storm track itself still represented a series of point locations at
every section of the track.

In (Lieberman-Cribbin et al. 2017), a study that looked at associations
between flooding and mental health outcomes after Hurricane Sandy struck
New York City, residents of areas that experience flooding completed
surveys and indicated their street address, city, and zip code. This
address information was then geocoded and matched up with the
appropriate latitude and longitude and represents data at at point
location.

Another example of a study that geocoded physical addresses to use point
location data for exposure analysis was (Brunkard, Namulanda, and Ratard
2008). This study analyzed mortality data in Louisiana prior to
Hurricane Katrina and characterize deaths related to the storm. Deaths
were mapped using the street location where death occurred which were in
turn matched with latitude and longitude coordinate systems (in cases
where the only address associated with a death was a nursing care
facility, the address of the nursing care facility was used).

Point locations represent the smallest spatial resolution of vector
data, and the specificity of this small scale can be appealing for
getting a close up look at individual human impacts of a tropical
cyclone event. However, there are reasons why this spatial scale is not
always used, even when it is accessible.

Physical exposure data from point locations is generally reliable and
consistent if it comes from monitors and weather stations, but in the
case of temporary networks of monitors, as mentioned in the previous
example from (East, Turco, and Mason Jr 2008), there isn't always enough
notice and lead time before a storm makes landfall for this to be a
reliable and consistent way to measure accurate data.

When using point location and individual level data on human impacts, an
important factor to consider is the preservation of privacy. Because
health data often contains highly sensitive information about
individuals, using point location data can compromise the privacy and
even safety of individuals if they are able to be traced to this
location. For this reason, even when point location and individual data
are available for study of human impacts, the data may be aggregated to
a larger spatial level (such as county or state) in order to preserve
the anonymity if the individuals.

A real life example to illustrate this point of privacy comes from the
Centers for Disease Control and Prevention, which collects data on many
different diseases and their outcomes. In the case of arboviral diseases
such as Zika, West Nile, Eastern equine encephalitis and many others,
data is provided at a spatial level no smaller than the county. Even so,
identifiable information such as age, sex, race, and ethnicity is
suppressed at the county level if less than three cases were reported in
the state in a given year. The main reason for this is to preserve
privacy and assure that individuals cannot be identified.

\hypertarget{polygons---zip-codecountyparishstate}{%
\subsubsection{Polygons - Zip
Code/County/Parish/State}\label{polygons---zip-codecountyparishstate}}

While point locations are the most common vectorized spatial scale that
physical exposure data are collected at, and human impacts data can also
be collected at that level, polygon vectors such as zip codes, counties,
parishes, and states tend to be more commonly used for human impacts
data. Information on human impacts such as birth outcomes,
hospitalizations, tax records, and demographic data are often recorded
at this spatial level.

It is important to remember that data at this spatial scale come from
individual and point location information at some point in the data
collection pipeline, but have been aggregated or deidentified in a way
that these invidual and point location data points are no longer
apparent. There are a variety of ways that this can be done such as
taking an aggregate measuremeant (for example the average birth weight
in an zip code), or as a count (such as the total number of
hospitalizations in a county). Reasons for making data available at this
spatial resolution are often practical (administrative organizations and
institutions often collect population level data at this level), but as
was mentioned previously it can also be used to preserve the privacy and
anonymity of individuals.

In (Huang, Rosowsky, and Sparks 2001), researchers created a damage
model from loss information (provided by a large insurer) to estimate
damage and losses in coastal zip codes of Florida and the Carolinas due
to Hurricanes Hugo and Andrew. The claims ratio (the total number of
claims in the code divided by the total number of insurance policies in
that zip code), and the damage ratio (the amount paid out by the insurer
divided by the total insured value) were compared to the maximum
gradient wind speed, obtained from wind field models. The relationships
between these ratios and wind speed suggested that when the mean wind
speed reached 20 meters per second, structural damage was more likely to
occur, and this damage became widespread when mean wind speed reached 30
meters per second (Huang, Rosowsky, and Sparks 2001).

Another example of physical exposures being assigned using polygon
spatial scales is FEMA disaster declarations. FEMA disaster declarations
are used to determine how much federal aid and funding needs to be
allocated to particular regions after a disaster, and in this way can be
used to estimate intensity of tropical cyclones and the resulting human
impacts on a given area. An example of a study that used this method to
assign exposure was (Horney et al. 2021) which looked at the impact of
natural disasters on this risk of suicide by identifying counties as
exposed if they had a single major disaster declaration between 2003 and
2005.

Another study that looked at human impacts in zip codes is (Lane et al.
2013). A review of the literature as well as using lessons from
Hurricane Sandy's impact on New York City, the researchers mapped
population vulnerability indicators in the 42 New York City United
Hospital Fund (UHF) neighborhoods, which are zip code aggregated areas
located within the city's five boroughs. In this study, these
neighborhoods were characterized by percentages such as percentage of
the neighborhood with delapidated or deteriorating housing, percentage
of neighborhood's residents living below the federal poverty line, the
percentage of neighborhood residents aged 85 or older, and the
percentage of residents with frequent mental distress.

Similar to a county is a parish (unique the state of Louisiana) and was
the spatial unit used in (Kinney et al. 2008), which looked at looked at
the prevalence of autism following prenatal exposure to tropical
cyclones. The study used maps of storm tracks provided by the National
Weather Service to identify the parishes that would be most intensely
impacted by the storm. A storm track passing through a parish's
boundaries as a proxy for the storm's intensity in that parish was one
of two characteristics that would be used to assign an exposure ranking
for residents of that parish (the other characterisitic being how
vulnerable to the effects of the storm the residents of the parish would
be). Ultimately this study concluded that in Louisiana the prevalence of
autism increased significantly in cohorts of children with prenatal
exposure to the tropical cyclones.

A study that used the county level to understand the human impacts of
tropical cyclones was (Parks et al. 2021). This study, which was
concerned with the rate of hospitalizations of older adults in response
to exposure to tropical cyclones, assigned the category of ``exposed''
to counties that experienced or exceed a gale force of greater than or
equal to 34 knots. The human impact of concern in this study was
quantified using data from enrollees from the Medicare cohort to
determine the cause of hospitalization and county of resident. Again, to
belabor an earlier point, the enrollees in this Medicare cohort are
individuals with residential addresses which could be considered as
point locations, that are instead aggregated at the county level to
preserve privacy. With the prior knowledge of which counties were
exposed to tropical the researchers were able to assign exposures to
these residents.

Spatial scales larger than point locations, zip codes, and counties are
the state and national levels of studying human impacts of tropical
cyclones. This spatial scale is not used as often but when it is it can
be useful to compare the emergency preparedness and policies of
different states. It can also reveal inequities in government response
to natural disasters.

For example, in (Willison et al. 2019), researchers quantified the
federal responses to Hurricanes Irma, Harvey, and Maria in Texas,
Florida, and Puerto Rico. They determined that in terms of federal
spending and staffing, Hurricane Maria in Puerto Rico was not responded
to in a manner commensurate with damage and need for aid compared to
Hurricanes Irma and Harvey in Texas and Puerto Rico.

Another study that used the spatial scale of state to study the human
impacts of tropical cyclones was (Grech and Scherb 2015), which showed
that in utero exposure to Hurricane Katrina (exposure assigned by
measuring the amount of rainfall after the storm in the Gulf states) had
an impact on the difference of survival of male and female fetuses,
which later impacted the male/female birth ratio at the end of the
pregnancy. Data on male and female live births was taken from the
Centers for Disease Control and Prevention on a monthly basis at the
state level from Alabama, Florida, Louisiana, and Mississippi. Rainfall
as a metric of exposure to Hurricane Katrina was also analyzed at the
state level, presented in inches for each state in the three days that
Hurricane Katrina struck the Gulf Coast.

\hypertarget{rasterized-data}{%
\subsection{Rasterized Data}\label{rasterized-data}}

Where vectorized geographic data objects can delineate boundaries and
precise locations on a map, rasterized data can fill in the space and
create a continous picture that shows variation across a geographic
space. Rasterized surfaces are created using gridded networks and
modeling.

For example, in (Anderson et al. 2020), rainfall was used to assess
exposure to tropical cyclones by summing hourly precipitation
measurements from the North American Land Data Assimilation System Phase
2 (NLDAS-2). This network spans the continental United States at 1/8∘
grid points. In this study the precipitation data from this grid was
used to create a daily precipitation total for each grid point, then the
gridpoints in a county were averaged together (based on counties' 1990
Census boundaries), which then created a daily county level estimate of
precipitation for each U.S. county from 1988-2011.

In (Anderson et al. 2020), a wind-based exposure metric was also used;
the county-level peak sustained surface wind during each storm.To do
this, the 1 minute surface wind was modeled at each county's mean
population center.

\hypertarget{temporal-scales}{%
\section{Temporal Scales}\label{temporal-scales}}

Thanks to scientitific institutions such as NOAA and the National
Weather Service, there are large networks of sensors and monitoring
equipment established across the United States that are capable of
recording physical exposure data at a fine temporal level. It is
possible to know the wind speed, amount of rainfall, and air temperature
at very fine temporal scales throughout the duration of a tropical
cyclone event. Human impacts data however, is typically not available at
such a fine scale, nor is such a scale sometimes even relevant. Whereas
physical exposure data may be collected in real time during the storm,
many of the human impacts that researchers are interested in may be only
known after the storm, and thus estimations may have to be made of what
happened in the past.

When thinking about temporal scales in tropical cyclone studies, it is
useful to think of them either as snapshots in time, or as cumulative
measurements up to a specific time. To illustrate this point, snapshots
in time can be thought of as thermometers. Thermometer measurements
represent the temperature at a specific point in time, and do no show
what the temperature was previously. In contrast to this, cumulative
measurements are the types of measurements taken by rain gauges. A rain
gauge collects and measures the amount of rain water gathered over a
specified duration of time, thus it is not a snapshot at all, but in
fact a cumulation. These two modes of thinking about temporal scales
will guide the way that we discuss the scales below.

\hypertarget{snapshots-of-time-the-thermometer-model}{%
\subsection{Snapshots of Time (The Thermometer
Model)}\label{snapshots-of-time-the-thermometer-model}}

Just as spatial scales exist at different levels (point location, zip
code, county, etc.), so do the temporal scales of time snapshots. The
difference here is that these time scales exist as frequency of
snapshots at the level of minutes, hours, days, etc. Here we discuss how
tropical cyclone studies utilize snapshots at different temporal scales,
because how often measurements are taken is important for understanding
what the data can and cannot tell us.

Relatively short intervals of time such as minute or hourly snapshots
are possible for many physical exposures because weather monitiors and
stations can easily record this information at consistent and regular
frequencies. Because the snapshots are taken at such regular intervals,
they can be used to construct continuous estimates of other storm
properties such as the storm track. The storm track itself is
constructed by stringing together measurements of wind speed and other
properties at a specific time and place. If a hurricane is tracked at
intervals of every 6 to 3 hours, this can also be interpolated to
estimate location at more regular intervals, for example every fifteen
minutes.

The sensors that were previously mentioned in (East, Turco, and Mason Jr
2008) are an example of a study that used this type of time scale. The
storm surge and inland flooding caused by Hurricane Ike were calculated
by these deployed pressure transducers which took measurements of surge
pressure, barometric pressure, and temperature every minute. The key
point here is that these snapshot measurements taken by the pressure
transducers represent the a property of the storm surge or flooding at a
single moment in time.

(S. Grabich et al. 2016) used storm track data from the National Oceanic
and Atmospheric's Association to compare disaster exposure assignment to
FEMA presidential disaster declarations. The data used to construct
NOAA's storm tracks records the magnitude and geographic location of the
storm every 6 hours, and each of these 6 hour measurements is a single
snapshot of where the storm center is in time and space. (Anderson et
al. 2020) also used storm tracks to generate a distance-based exposure
metric, and they interpolated the tracks to get storm track locations at
every 15 minutes. Even though these interpolated values are estimates
created from a model, they still represent snapshots of the time and
location of the storm center, but at a much finer scale.

Measurements of human impacts are not available as often at such fine
temporal scales, although there are certainly certain measurements that
could be taken, for example the timing of phone calls to emergency
departments or more recently by social media posts that pertain to the
storm.

Some snapshot measurements are taken at intervals that are much larger
in temporal scop than minutes or hours, for example on the time scale of
days and weeks. Because physical exposure data is so easily recorded at
fine temporal resolutions, snapshots that are taken at more daily or
weekly frequencies are often associated more with human impacts. Daily
or weekly summaries of abseentism at work or school due to inclement
weather conditions are an example of snapshot data taken at this level.

(Parks et al. 2021) is an example of a study that took daily
measurements to study human impacts of tropical cyclones, in this case
daily hospitalization rates. The study used a conditional quasi-Poisson
regression model to analyze the daily hospitalization rate up to 7 days
after the day of hurricane exposure. Hospitalizations from respiratory
diseases and from injuries increased all days after the day of exposure,
peaking on the first and second days respectively. For several other
causes of hospitalization such as cardiovascular disease, endocrine
disorders, genitourinary diseases, infectious and parasitic diseases,
nervous system diseases, and skin and subcutaneous tissue diseases, the
rate of hospitalization decreased on the day of exposure to the
hurricane and then peaked about 1 to 3 days later before returning to
the expected rate before the storm (Parks et al. 2021). The daily number
of hospitalizations represents a snapshot because it is a slice of time
and doesn't tell us about the total and final number of hospitalizations
resulting from the entire storm.

Zooming out to even broader time scales, snapshots can be taken at
seasonal, yearly, or even decadal frequencies. Once again this will also
be associated more strongly with human impacts. For example, the U.S.
Census occurs every 10 years and is acts as a proxy for a snapshot of
the nation's population. These 10 year snapshots can reveal changes in
population, socioeconomic status, and other information in regions
affected by tropical cyclone events.

In the study (Jaycox et al. 2010), mental health outcomes of New Orleans
school children was assessed 15 months after Hurricane Katrina, and an
intervention strategy was implemented to help treat these outcomes. The
children were assessed for symptom of post traumatic stress disorder and
depression first at baseline (December 2006/January 2007), again 5
months after intervention, and then again 10 months after intervention.
Overall the treatment was shown to reduce symptoms of PTSD, although
this was not even across all groups.

\textbf{\emph{Try to think of an example of snapshots that would be
taken on an annual basis}}

\hypertarget{cumulative-measures-over-time-the-rain-gauge-model}{%
\subsubsection{Cumulative Measures Over Time (The Rain Gauge
Model)}\label{cumulative-measures-over-time-the-rain-gauge-model}}

Cumulative measurements within a defined period of time give different
information about exposures and impacts of tropical cyclones. Instead of
giving information about a precise moment, cumulative measurements
inform researchers on the amassed value of a specified measurement, in
other words everything up to a certain point. Rain gauges are an
effective way to visualize cumulative data because the measurement
reflects the total amount of rain water that has been collected in a
designated period of time. The final volume of the rain gauge is not a
snapshot, as an specific quantity like 3 inches of rain doesn't exist
all at once, but instead it shows the additive amount that has occurred
in a known time frame.

Similar to the way that snapshot measurements can be recorded at varying
intervals of time, cumulative data can represent total measurements over
varying spans of time. Cumulative measurements of physical exposures
will often be taken over smaller periods of time, on the order of
minutes, hours, or days which corresponds to the lengths of the storms
themselves. Over a the span of several hours or several days such
measurements as the total amount of rainfall, or the number of minutes
where wind exceeded a certain speed could come to represent cumulative
measurements on a smaller scale. Cumulative measurements that give
information on human impacts typically happen on the order of days,
weeks, months or years. Daily and weekly counts of deaths and
hospitalizations in an area due to the effects of tropical cyclones are
cumulative measurements. Over the span of months and years, cumulative
measurements of human impacts could encompass things like the total
dollar value of insurance claims in a particular neighborhood struck by
flooding, or the net migration of people in or out of a community after
the impact of a storm.

Many cumulative measurements are taken of human impacts of tropical
cyclones simply because there is no way to measure certain impacts at a
precise moment or in the midst of the storm, so damages and impacts have
to be assessed once the storm has passed. It is also important to note
that snapshots and cumulative measurements are not always so clearly
separated. A daily count of hospitalizations could reflect the
cumulative number of patients hospitalized (imagining them all in a rain
gauge together), or it could represent a snapshot of how many people
have been hospitalized at a precise moment.

\hypertarget{implications-of-not-improving-this-integration}{%
\section{Implications of not improving this
integration}\label{implications-of-not-improving-this-integration}}

Temporal and spatial misalignment poses certain challenges to
researchers investigating the human impacts of tropical cyclones.
Integrating data at different spatial scales is often accomplished by
aggregating one set of data to match the data that is at a greater
scale. Sometimes however, misalignment exists not at different scales,
but at the same scale in different places, such as a residence that is
miles away from the nearest weather monitor. In cases such as this,
physical exposure is assigned by matching residences or addresses to the
nearest monitor, or an interpolation model will be created to estimate
exposures. In the following sections we will discuss aggregation and
some of the implications that arise from it. We will then describe
matching and interpolating and similary describe the implications that
result from these methods.

\hypertarget{aggregating-to-integrate-data-at-different-scales}{%
\subsection{Aggregating to Integrate Data at Different
Scales}\label{aggregating-to-integrate-data-at-different-scales}}

Aggregating physical exposures is what researchers do by assigning a
single exposure value to a wider spatial area, such as a zip code or
county. Since the human impacts data available will often be at this
scale anyways, the finer physical exposure data will be generalized to
this level as well. This can be done in a number of ways, such as taking
a wind speed measurement from a monitor at the center of the county.
This wind exposure value will then be assigned to the entire county.
Windspeed at the moment a tropical cyclone makes landfall is another
exposure assignment method, as was done in (Shao et al. 2017), in which
the maximum wind speed when a tropical cyclone made landfall in a
particular coastal county was used to assign that particular county's
exposure. Cumulativie rainfall in the entire county, distance of the
county from the storm track, number of tornadoes in the county, and
floodindg in the county are other methods of assigning an aggregate
exposure value to a certain region.

It is often the case that these aggregated values will be determined to
categorize a county or other spatial area as exposed or unexposed, based
on some kind of threshold. In (Parks et al. 2021), researchers
considered counties exposed if the peak sustained wind that day exceeded
a gale force greater than or equal to 34 knots on the Beaufort scale
when the cyclone was at the point of closest approach to the county.
Entire counties will also be categorized as unexposed or exposed based
on whether the storm track of the tropical cyclone passed through their
borders or not.

The Saffir-Simpson scale is an example of how entire storms are often
classified by their maximum wind speed. Forecasters classify hurricanes
into categories on the Saffir-Simpson scale based on maximum sustained
surface wind speed. This is defined as the peak one minute wind speed at
a height of 10 feet over an unobstructed exposure (Taylor et al. 2010).
The Saffir-Simpson scale uses five different bins to classify varying
levels of wind speed and determine the severity of a storm. The first
level, Category 1 is designated for hurricanes and tropical storms with
maximum wind speeds of between 64 - 82 knots and is generally considered
dangerous to people, livestock, and pets from the hazard of flying and
falling debris (Taylor et al. 2010). On the higher end of the scale,
Category 5 designates hurricanes with maximum wind speeds above 137
knots and is considered to have catastrophic effect on damage and a high
probability of injury or death to people, livestock, and pets even if
they are sheltering indoors (Taylor et al. 2010). An important
limitation of the Saffir-Simpson scale is that it doesn't account for
other hurricane-related impact variables such as storm surges, flooding,
and tornadoes (Taylor et al. 2010). (Shao et al. 2017) used this scale
to assign wind speed categories to counties along the Gulf Coast in a
study assessing perceptions of risk to tropical cyclones. (Belasen and
Polachek 2008) used this scale as well in a study that compared
hurricane intensities to average earnings in different counties.
Hurricanes with categories one, two or three were considered lower
intensity, and hurricanes of counties four and five were considered high
intensity.

\hypertarget{implications-of-aggregating-data}{%
\subsubsection{Implications of Aggregating
Data}\label{implications-of-aggregating-data}}

When researchers have physical exposure data at a very fine resolution,
perhaps even continuous, and human impacts data at a more aggregate
level, it is common and practical to aggregate the physical exposure
data. When dealing with aggregated data of any kind, it is important to
realize that information on the individual level is lost. Researchers
should be mindful when aggregating data, particularly continuous data,
that they are losing information. In this section we will discuss some
of the major implications of aggregating data which include ecological
bias, misclassification and measurement error, and the process of
categorizing continuous data.

\hypertarget{ecological-bias}{%
\subsubsection{Ecological Bias}\label{ecological-bias}}

Physical exposures to tropical cyclones, as well as human impacts, are
observed across spatial gradiants, and though specific point locations
may exist for a weather monitor recording maximum wind speed or
rainfall, that point location data is often aggregated to a larger
spatial unit. For example, several weather monitors at specific point
locations recording maximum wind speed may be replaced by the maximum
wind speed at the center of the county. Data such as this is known as
ecological data, aggregate data, or contextual-level data. Studies that
use such data are known as ecological studies (Sedgwick 2014).
Ecological bias occurs whenever the aggregate association between an
exposure and an outcome does not properly reflect the association on the
individual level (Greenland and Morgenstern 1989). There are times when
this is not a concern, such as when the aggregate is a count. For
example in (Zahran, Tavani, and Weiler 2013), the daily casualty count
was reported for individual counties in the Southern United States,
using count data from the Spatial Hazard Events and Losses Database.
However, when estimates derived from ecological studies are used to
infer individual estimates, ecological bias will likely be present.
Especially when there is heterogeneity present in an aggregated
population, an ecological estimate should not be taken to be
representative of individual estimates.

\hypertarget{misclassification-and-measurement-error-in-aggregating-data}{%
\subsubsection{Misclassification and Measurement Error in Aggregating
Data}\label{misclassification-and-measurement-error-in-aggregating-data}}

When aggregating data, another concern that arises is misclassification
or measurerement error. Misclassification error occurs when exposure and
outcome variables are measured in categories and the wrong category is
assigned to a particular case/observation - for example when a case that
is exposed is incorrectly categorized as unexposed. Failure to classify
exposure accurately(for example, classifying certain observations as
exposed to a storm when they really were not, or vice-versa), allows
misclassification bias to move the results of the study further from the
true parameter. Measurement error occurs when the variables being
measured are continuous, such as the amount of precipitation or the wind
speed that was measured during a tropical cyclone.

Environmental epidemiology studies are often prone to misclassification
error because the methods of assessing exposure are not always congruent
with the way that researchers conduct human impact studies. It is easy
to map the path of a tropical cyclone's center, and categorize every
county it passes through as an exposed county. However, this information
by itself would not give the researcher any information about population
centers that the storm track got close to. A town within an exposed
county may or may not have been close to the storm's path. Conversely,a
town in an unexposed county could be located very close to the border of
an exposed county, and even be closer to the storm's track than a
different town within that exposed county. An example of a study that
could be prone to this kind of bias is (Kinney et al. 2008) where
Louisiana parishes were considered vulnerable to hurricane exposure
based on whether or not the storm center passed through that parish. It
is possible that the cases considered exposed based on living in these
parishes were not in fact exposed since the storm may have passed
through only a certain part of the parish. Never the less, all cases in
a parish are considered exposed or unexposed in the aggregate.

\#\#\#\#{[}Need to figure a way to conclude this section{]}\#\#\#\#

\hypertarget{when-data-have-the-same-scale-but-are-at-different-locations}{%
\subsection{When Data Have the Same Scale but are at Different
Locations}\label{when-data-have-the-same-scale-but-are-at-different-locations}}

Sometimes, researchers may have access to data that is down to the point
source, both for physical exposures and also for human impacts. Very
likely however, these point sources will not be the same. Here the issue
is not of integrating different resolution levels, but rather of
matching different point locations. Weather monitoring stations may be
set up regularly in a geographic region, sometimes in regular grids, but
the human impacts point locations could be tied to a single residential
address that is likely not at a weather station or in a grid.

One way to resolve this spatial misalignment is to assign exposure to
the residential addresses based on the closest weather monitoring
station (Kim, Sheppard, and Kim 2009). This is a common method that is
employed in many other areas of environmental epidemiology, including
studies on the impacts of wildfire smoke plumes and urban smog on
respiratory health. Typically, a distance threshold will be determined
for a monitoring station or a sensor, and any residence within that
distance will be assigned an exposure value from that monitoring point.

There are several drawbacks to assigning exposure to human residences
based on distance from a weather monitoring stations. One of them is
that in more rural areas, in situ observations may be sparse and this
limits the information between monitors, and diminishes the accuracy of
exposures assigned to individuals located between those monitors (Gan et
al. 2017).The exposed population is also reduced when you rely on
distance from monitoring sites, because you can only include individuals
who are close enough to reasonably be assigned the exposure from that
site (Lassman et al. 2017). This can be a problem, as large populations
are often required for detection of health impacts. Depending on the
exposure of interest, topography, climate, and localized weather
patterns will also render sites beyond a limited threshold distance from
the site as unrealistic to be assigned the value from the monitoring
station.

Another method of assigning exposure to spatially misaligned individuals
is to interpolate, using models. Spatial interpolation is the prediction
of values or metrics of specific points within a defined region based on
some sort of spatial model (Li and Heap 2014). Kriging is one such
method that creates continuous spatial surfaces for understanding
environmental variables like air pollution, minerals, soil, and
meteorological conditions (Liang and Kumar 2013). It is a type of
Generalized Least Square Regression Algorithm (Li and Heap 2014). This
method has been used extensively in modeling the effects of air
pollution in places like California, as was done in (Kim, Sheppard, and
Kim 2009) to look at air pollution exposure in Los Angeles, California.
The study utilized a kriging model and as well as using the nearest
weather monitoring station to assign air pollution exposure to
residential locations in Los Angeles. Kriging is widely applicable to
studies of tropical cyclones as well. Researchers in South Carolina used
kriging interpolation to analyse rainfall data and create
spatio-temporal model in 2015 during a particularly strong storm season.

A limitation of both matching physical exposure monitors to residences,
and using interpolation models to infer exposure values, is that
meteorological events can damage these monitors. Strong storm events
that produce high enough winds can blow away or damage senors. If
precipitation is being measured, rain that comes down in slants can also
make measurements less accurate. These are issues that are particularly
concerning for ground based monitors.

The more complicated that a model becomes, the harder it is to
interpret. This is why the most simple method that can be utilized
(sometimes simply matching human point locations to the nearest monitor)
is the best way to go. Some exposures, like wind speed, are relatively
homogenous over large areas, so the different methods will not give much
variation in results.

\hypertarget{misclassification-for-same-scale-different-locations}{%
\subsubsection{Misclassification for Same Scale Different
Locations}\label{misclassification-for-same-scale-different-locations}}

The obvious goal of assigning tropical cyclone exposures to individual
point locations by matching values from the nearest monitoring site or
spatially interpolating, is to estimate exposure values accuarately.
This is crucial to avoid exposure misclassification. The more spatially
heterogeneous that an environmental exposure is, the more room there is
for exposure misclassification to occur. In studies of wildfires and air
pollution, concentrations of PM2.5 and other air pollutants can very
greatly within relatively small spatial areas. Other factors like
windspeed and rainfall however, are fairly homogenous across spatial
areas. This means that interpolating and assigning exposure based on the
nearest monitoring sites may result in less exposure
misclassificationhave. Tornadoes on the other hand tend to be very
localized and can easily be overlooked with these methods.

When assigning exposure to an individual point location based on the
nearest monitoring site, the further this location is from the
monitoring site, the more likely it is that the monitoring site won't
reflect an exposure estimate accurately. Topography, complicated weather
patterns, and other things could complicate this measurement.

When interpolating, the environmental exposure of concern will partially
determine the potential for misclassification. Using the examples from
above of windspeed and rainfall, it is unlikely that much
misclassification would occur over a spatial interface since they are
homogenous over large areas.

\hypertarget{conclusiondiscussion}{%
\section{Conclusion/Discussion}\label{conclusiondiscussion}}

The aim of this paper has been to show that studying the human impacts
of tropical cyclones requires use of multiple datasets from different
sources, at different temporal and spatial scales, and resulting from
different data collection methods. We've shown how physical exposure
data from tropical cyclones usually comes from point location monitors
and sensors such as rain and stream gauges, wind vanes, at weather
stations that are often a part of large scale networks operated by NOAA
and the NWS. We've shown how in contrast to this, that data on human
impacts comes from administrative sources at hospitals, schools, and
governments, often at the level of the zip code or county. These
differences in temporal and spatial scale make use of the data difficult
for researchers to immediately utilize. Nevertheless, several methods
exist for integrating these datasets, and we've briefly described
aggregation, interpolation, and matching as the main methdos for doing
this. Though useful, these methods are not without their limitation,
such as their propensity towards ecological bias and exposure
misclassification.

Ecological bias and exposure misclassification are a problem, because
they impact the external validity of tropical cyclone studies. If not
adequately mitigated, they can produce measures of association that do
not reflect the true impact that tropical cyclones have on coastal
communities. In other words, these kinds of bias and error make tropical
cyclone studies less generalizable, which renders them less effective in
predicting for future events and preparing for them. That leaves large
gaps of uncertainty when it comes to creating resilient communities that
can withstand tropical cyclone events, and leaves more vulnerable
populations and communities at an elevated risk. {[}Cite Daniel's
paper{]}

There is reason to be optimistic that we can harness the power of large
datasets on physical, and increasing precision in model building, to
understand and predict the effects of tropical cyclones on human health,
economic viability, infrastructure, and social dynamics. This is of
course contingent on our ability as researchers to make sure that the
data on human impacts is of a similar quality and translatable
resolution as the high quality data we already have on physical
exposures. With larger populations expected to live in areas threatened
by tropical cyclones in the future, this is an extremely important
undertaking for interdisciplinary teams of meteorologists,
epidemiologists, economists, and diaster planners.

Link to online book Geocomputation in R
\url{https://geocompr.robinlovelace.net/} \# References

\hypertarget{refs}{}
\leavevmode\hypertarget{ref-ahrens2005essentials}{}%
Ahrens, CD. 2005. ``Essentials of Meteorology Essentials of Meteorology:
An Invitation to the Atomosphere.'' Thomson Brooks/Cole Calif.

\leavevmode\hypertarget{ref-anderson2020assessing}{}%
Anderson, G Brooke, Joshua Ferreri, Mohammad Al-Hamdan, William Crosson,
Andrea Schumacher, Seth Guikema, Steven Quiring, Dirk Eddelbuettel,
Meilin Yan, and Roger D Peng. 2020. ``Assessing United States
County-Level Exposure for Research on Tropical Cyclones and Human
Health.'' \emph{Environmental Health Perspectives} 128 (10): 107009.

\leavevmode\hypertarget{ref-aschengrau2013essentials}{}%
Aschengrau, Ann, and George R Seage. 2013. \emph{Essentials of
Epidemiology in Public Health}. Jones \& Bartlett Publishers.

\leavevmode\hypertarget{ref-belasen2008hurricanes}{}%
Belasen, Ariel R, and Solomon W Polachek. 2008. ``How Hurricanes Affect
Wages and Employment in Local Labor Markets.'' \emph{American Economic
Review} 98 (2): 49--53.

\leavevmode\hypertarget{ref-bevilacqua2020understanding}{}%
Bevilacqua, Kristin, Rehana Rasul, Samantha Schneider, Maria Guzman,
Vishnu Nepal, Deborah Banerjee, Joann Schulte, and Rebecca M Schwartz.
2020. ``Understanding Associations Between Hurricane Harvey Exposure and
Mental Health Symptoms Among Greater Houston-Area Residents.''
\emph{Disaster Medicine and Public Health Preparedness}, 1--8.

\leavevmode\hypertarget{ref-brunkard2008hurricane}{}%
Brunkard, Joan, Gonza Namulanda, and Raoult Ratard. 2008. ``Hurricane
Katrina Deaths, Louisiana, 2005.'' \emph{Disaster Medicine and Public
Health Preparedness} 2 (4): 215--23.

\leavevmode\hypertarget{ref-deryugina2017fiscal}{}%
Deryugina, Tatyana. 2017. ``The Fiscal Cost of Hurricanes: Disaster Aid
Versus Social Insurance.'' \emph{American Economic Journal: Economic
Policy} 9 (3): 168--98.

\leavevmode\hypertarget{ref-deryugina2018economic}{}%
Deryugina, Tatyana, Laura Kawano, and Steven Levitt. 2018. ``The
Economic Impact of Hurricane Katrina on Its Victims: Evidence from
Individual Tax Returns.'' \emph{American Economic Journal: Applied
Economics} 10 (2): 202--33.

\leavevmode\hypertarget{ref-east2008monitoring}{}%
East, Jeffery W, Michael J Turco, and Robert R Mason Jr. 2008.
``Monitoring Inland Storm Surge and Flooding from Hurricane Ike in Texas
and Louisiana, September 2008.'' \emph{Surge} 29: 95--20833.

\leavevmode\hypertarget{ref-ferdinand2005hurricane}{}%
Ferdinand, Keith C. 2005. ``The Hurricane Katrina Disaster: Focus on the
Hypertensive Patient.'' \emph{The Journal of Clinical Hypertension} 7
(11): 679--80.

\leavevmode\hypertarget{ref-gan2017comparison}{}%
Gan, Ryan W, Bonne Ford, William Lassman, Gabriele Pfister, Ambarish
Vaidyanathan, Emily Fischer, John Volckens, Jeffrey R Pierce, and Sheryl
Magzamen. 2017. ``Comparison of Wildfire Smoke Estimation Methods and
Associations with Cardiopulmonary-Related Hospital Admissions.''
\emph{GeoHealth} 1 (3): 122--36.

\leavevmode\hypertarget{ref-grabich2016measuring}{}%
Grabich, SC, J Horney, C Konrad, and DT Lobdell. 2016. ``Measuring the
Storm: Methods of Quantifying Hurricane Exposure with Pregnancy
Outcomes.'' \emph{Natural Hazards Review} 17 (1): 06015002.

\leavevmode\hypertarget{ref-grabich2016hurricane}{}%
Grabich, Shannon C, Whitney R Robinson, Stephanie M Engel, Charles E
Konrad, David B Richardson, and Jennifer A Horney. 2016. ``Hurricane
Charley Exposure and Hazard of Preterm Delivery, Florida 2004.''
\emph{Maternal and Child Health Journal} 20 (12): 2474--82.

\leavevmode\hypertarget{ref-grech2015hurricane}{}%
Grech, Victor, and Hagen Scherb. 2015. ``Hurricane Katrina: Influence on
the Male-to-Female Birth Ratio.'' \emph{Medical Principles and Practice}
24 (5): 477--85.

\leavevmode\hypertarget{ref-greenland1989ecological}{}%
Greenland, Sander, and Hal Morgenstern. 1989. ``Ecological Bias,
Confounding, and Effect Modification.'' \emph{International Journal of
Epidemiology} 18 (1): 269--74.

\leavevmode\hypertarget{ref-horney2021impact}{}%
Horney, Jennifer A, Ibraheem M Karaye, Alexander Abuabara, Sera
Gearhart, Shannon Grabich, and Maria Perez-Patron. 2021. ``The Impact of
Natural Disasters on Suicide in the United States, 2003--2015.''
\emph{Crisis: The Journal of Crisis Intervention and Suicide Prevention}
42 (5): 328.

\leavevmode\hypertarget{ref-huang2001long}{}%
Huang, Zhigang, David V Rosowsky, and Peter R Sparks. 2001. ``Long-Term
Hurricane Risk Assessment and Expected Damage to Residential
Structures.'' \emph{Reliability Engineering \& System Safety} 74 (3):
239--49.

\leavevmode\hypertarget{ref-jaycox2010children}{}%
Jaycox, Lisa H, Judith A Cohen, Anthony P Mannarino, Douglas W Walker,
Audra K Langley, Kate L Gegenheimer, Molly Scott, and Matthias Schonlau.
2010. ``Children's Mental Health Care Following Hurricane Katrina: A
Field Trial of Trauma-Focused Psychotherapies.'' \emph{Journal of
Traumatic Stress: Official Publication of the International Society for
Traumatic Stress Studies} 23 (2): 223--31.

\leavevmode\hypertarget{ref-kim2009health}{}%
Kim, Sun-Young, Lianne Sheppard, and Ho Kim. 2009. ``Health Effects of
Long-Term Air Pollution: Influence of Exposure Prediction Methods.''
\emph{Epidemiology}, 442--50.

\leavevmode\hypertarget{ref-kinney2008autism}{}%
Kinney, Dennis K, Andrea M Miller, David J Crowley, Emerald Huang, and
Erika Gerber. 2008. ``Autism Prevalence Following Prenatal Exposure to
Hurricanes and Tropical Storms in Louisiana.'' \emph{Journal of Autism
and Developmental Disorders} 38 (3): 481--88.

\leavevmode\hypertarget{ref-lane2013health}{}%
Lane, Kathryn, Kizzy Charles-Guzman, Katherine Wheeler, Zaynah Abid,
Nathan Graber, and Thomas Matte. 2013. ``Health Effects of Coastal
Storms and Flooding in Urban Areas: A Review and Vulnerability
Assessment.'' \emph{Journal of Environmental and Public Health} 2013.

\leavevmode\hypertarget{ref-lassman2017spatial}{}%
Lassman, William, Bonne Ford, Ryan W Gan, Gabriele Pfister, Sheryl
Magzamen, Emily V Fischer, and Jeffrey R Pierce. 2017. ``Spatial and
Temporal Estimates of Population Exposure to Wildfire Smoke During the
Washington State 2012 Wildfire Season Using Blended Model, Satellite,
and in Situ Data.'' \emph{GeoHealth} 1 (3): 106--21.

\leavevmode\hypertarget{ref-li2014spatial}{}%
Li, Jin, and Andrew D Heap. 2014. ``Spatial Interpolation Methods
Applied in the Environmental Sciences: A Review.'' \emph{Environmental
Modelling \& Software} 53: 173--89.

\leavevmode\hypertarget{ref-li2018novel}{}%
Li, Zhenlong, Cuizhen Wang, Christopher T Emrich, and Diansheng Guo.
2018. ``A Novel Approach to Leveraging Social Media for Rapid Flood
Mapping: A Case Study of the 2015 South Carolina Floods.''
\emph{Cartography and Geographic Information Science} 45 (2): 97--110.

\leavevmode\hypertarget{ref-liang2013time}{}%
Liang, Dong, and Naresh Kumar. 2013. ``Time-Space Kriging to Address the
Spatiotemporal Misalignment in the Large Datasets.'' \emph{Atmospheric
Environment} 72: 60--69.

\leavevmode\hypertarget{ref-lieberman2017self}{}%
Lieberman-Cribbin, Wil, Bian Liu, Samantha Schneider, Rebecca Schwartz,
and Emanuela Taioli. 2017. ``Self-Reported and Fema Flood Exposure
Assessment After Hurricane Sandy: Association with Mental Health
Outcomes.'' \emph{PLoS One} 12 (1): e0170965.

\leavevmode\hypertarget{ref-lovelace2019geocomputation}{}%
Lovelace, Robin, Jakub Nowosad, and Jannes Muenchow. 2019.
\emph{Geocomputation with R}. Chapman; Hall/CRC.

\leavevmode\hypertarget{ref-parks2021tropical}{}%
Parks, Robbie M, G Brooke Anderson, Rachel C Nethery, Ana Navas-Acien,
Francesca Dominici, and Marianthi-Anna Kioumourtzoglou. 2021. ``Tropical
Cyclone Exposure Is Associated with Increased Hospitalization Rates in
Older Adults.'' \emph{Nature Communications} 12 (1): 1--12.

\leavevmode\hypertarget{ref-scaramutti2019mental}{}%
Scaramutti, Carolina, Christopher P Salas-Wright, Saskia R Vos, and Seth
J Schwartz. 2019. ``The Mental Health Impact of Hurricane Maria on
Puerto Ricans in Puerto Rico and Florida.'' \emph{Disaster Medicine and
Public Health Preparedness} 13 (1): 24--27.

\leavevmode\hypertarget{ref-sedgwick2014ecological}{}%
Sedgwick, Philip. 2014. ``Ecological Studies: Advantages and
Disadvantages.'' \emph{Bmj} 348.

\leavevmode\hypertarget{ref-shao2017understanding}{}%
Shao, Wanyun, Siyuan Xian, Barry D Keim, Kirby Goidel, and Ning Lin.
2017. ``Understanding Perceptions of Changing Hurricane Strength Along
the Us Gulf Coast.'' \emph{International Journal of Climatology} 37 (4):
1716--27.

\leavevmode\hypertarget{ref-taylor2010saffir}{}%
Taylor, Harvey Thurm, Bill Ward, Mark Willis, and Walt Zaleski. 2010.
``The Saffir-Simpson Hurricane Wind Scale.'' \emph{Atmospheric
Administration: Washington, DC, USA}.

\leavevmode\hypertarget{ref-willison2019quantifying}{}%
Willison, Charley E, Phillip M Singer, Melissa S Creary, and Scott L
Greer. 2019. ``Quantifying Inequities in Us Federal Response to
Hurricane Disaster in Texas and Florida Compared with Puerto Rico.''
\emph{BMJ Global Health} 4 (1).

\leavevmode\hypertarget{ref-yan2020tropical}{}%
Yan, M, G Anderson, A Wilson, F Dominici, Y Wang, M Al-Hamdan, W
Crosson, et al. 2020. ``Tropical Cyclone Exposure and Risk of Emergency
Medicare Hospital Admissions for Cardiorespiratory Diseases in 175
United States Counties, 1999-2010.'' \emph{Epidemiology} In Press.

\leavevmode\hypertarget{ref-zahran2013daily}{}%
Zahran, Sammy, Daniele Tavani, and Stephan Weiler. 2013. ``Daily
Variation in Natural Disaster Casualties: Information Flows, Safety, and
Opportunity Costs in Tornado Versus Hurricane Strikes.'' \emph{Risk
Analysis} 33 (7): 1265--80.

\end{document}

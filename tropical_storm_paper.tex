\documentclass[]{article}
\usepackage{lmodern}
\usepackage{amssymb,amsmath}
\usepackage{ifxetex,ifluatex}
\usepackage{fixltx2e} % provides \textsubscript
\ifnum 0\ifxetex 1\fi\ifluatex 1\fi=0 % if pdftex
  \usepackage[T1]{fontenc}
  \usepackage[utf8]{inputenc}
\else % if luatex or xelatex
  \ifxetex
    \usepackage{mathspec}
  \else
    \usepackage{fontspec}
  \fi
  \defaultfontfeatures{Ligatures=TeX,Scale=MatchLowercase}
\fi
% use upquote if available, for straight quotes in verbatim environments
\IfFileExists{upquote.sty}{\usepackage{upquote}}{}
% use microtype if available
\IfFileExists{microtype.sty}{%
\usepackage[]{microtype}
\UseMicrotypeSet[protrusion]{basicmath} % disable protrusion for tt fonts
}{}
\PassOptionsToPackage{hyphens}{url} % url is loaded by hyperref
\usepackage[unicode=true]{hyperref}
\hypersetup{
            pdftitle={Challenges of Integrating Physical Exposure and Human Impacts Data in Tropical Cyclone Studies},
            pdfauthor={Matthew Hughes and Brooke Anderson},
            pdfborder={0 0 0},
            breaklinks=true}
\urlstyle{same}  % don't use monospace font for urls
\usepackage[margin=1in]{geometry}
\usepackage{graphicx,grffile}
\makeatletter
\def\maxwidth{\ifdim\Gin@nat@width>\linewidth\linewidth\else\Gin@nat@width\fi}
\def\maxheight{\ifdim\Gin@nat@height>\textheight\textheight\else\Gin@nat@height\fi}
\makeatother
% Scale images if necessary, so that they will not overflow the page
% margins by default, and it is still possible to overwrite the defaults
% using explicit options in \includegraphics[width, height, ...]{}
\setkeys{Gin}{width=\maxwidth,height=\maxheight,keepaspectratio}
\IfFileExists{parskip.sty}{%
\usepackage{parskip}
}{% else
\setlength{\parindent}{0pt}
\setlength{\parskip}{6pt plus 2pt minus 1pt}
}
\setlength{\emergencystretch}{3em}  % prevent overfull lines
\providecommand{\tightlist}{%
  \setlength{\itemsep}{0pt}\setlength{\parskip}{0pt}}
\setcounter{secnumdepth}{0}
% Redefines (sub)paragraphs to behave more like sections
\ifx\paragraph\undefined\else
\let\oldparagraph\paragraph
\renewcommand{\paragraph}[1]{\oldparagraph{#1}\mbox{}}
\fi
\ifx\subparagraph\undefined\else
\let\oldsubparagraph\subparagraph
\renewcommand{\subparagraph}[1]{\oldsubparagraph{#1}\mbox{}}
\fi

% set default figure placement to htbp
\makeatletter
\def\fps@figure{htbp}
\makeatother


\title{Challenges of Integrating Physical Exposure and Human Impacts Data in
Tropical Cyclone Studies}
\author{Matthew Hughes and Brooke Anderson}
\date{May 25, 2020}

\begin{document}
\maketitle

\section{Introduction}\label{introduction}

Tropical cyclones---which encompasses hurricanes as well as tropical
storms and tropical depressions--- regularly threaten coastal
communities across the Eastern and Southern United States. From 2000 to
2019, tropical cyclones cost the United States at least 811 billion
dollars in damages({\textbf{???}} billion dollar disasters). Tropical
cyclones in that same time frame resulted in 6,010 human fatalities,
averaging 301 deaths per year ({\textbf{???}} billion dollar disasters).
Tropical cyclones upset coastal communities and society by damaging
property, disrupting local economies, and harming human health. This is
why they are so critical to study.

Researchers have observed that in utero exposure to tropical cyclones
leads to adverse birth outcomes. (Kinney et al. 2008) observed higher
rates of autism in children born to mothers who had higher rates of
storm exposure than children born to mothers who were exposed to later
intensities. This is not very surprising when one considers that
tropical storms are highly stressful events, and stress during a
pregnancy is known to have strong impacts on the developing fetus. The
scientific literature also reveals evidence of mental health outcomes
associated with populations exposed to tropical cyclones. Survivors of
tropical storms often report higher levels of depression, anxiety, and
PTSD, due to reduced access to important medical and social services,
property damages, poor sanitation, and displacement after storms.
(Lieberman-Cribbin et al. 2017) found higher levels of PTSD in New York
City residents who were exposed to flooding after Hurricane Sandy.
Beyond health impacts, both mental and physical, tropical storms create
incredible strains on the economies of the Southeastern United States.
The average cost of a tropical cyclone event in the US is 21.2 billion
per event, CPI-adjusted ({\textbf{???}} billion dollar disasters).

Clearly there are severe human impacts that result from exposure to
tropical storms. Tropical cyclones represent an environmental health
threat that is not going to disappear, and given that coastal regions of
the Southeastern US are experiencing population growth, it is likely
that higher numbers of people will be put at risk in the future.
Avoiding these risks is not possible, but building resilience in
communities after they experience tropical cyclone events is key to
mitigating damages and preparing for future disasters. Creating lasting
and resilient communities in areas prone to tropical cyclones requires
that researchers understand which populations and locations are at the
greatest risk for negative exposures to tropical storms. This requires
data that allows researchers to assess where in space and time tropical
storms occur, and also where in space and time individuals and
populations are experiencing impacts from these storms.

Lots of researchers are exploring this using multidisciplinary teams,
however a key challenge is integrating data from across disciplines. For
example: extensive physical exposure data is often available for
tropical cyclones as they near and cross communities in the United
States. This data can come both from established monitoring networks,
like {[}NOAA network name?{]}, but may also result from data collection
efforts during or after the storm by atmospheric scientists and
engineers seeking to characterize a storm. Researchers studying the
human impacts of these storms, including epidemiologists, economists,
and social scientists are interested in this data as well, but it is
often unavailable in a way that is accessible for them to use. Resolving
physical exposure and human impact datasets is challenging because the
human impact data and physical exposure data do not have congruent
resolutions.

Here we explore cases and implications of integrating data at different
temporal and spatial scales, focusing as an example on human impact
studies of tropical cyclones in the US. We begin by investigating the
reasons that spatial and temporal misaligment exist in the study of
tropical cyclones. We then describe the main spatial and temporal scales
used, and finally assess some of the consequences that result from
integrating physical exposure data with human impacts data.

\section{Spatial and Temporal Misalignment: Origins of Integration
Challenges}\label{spatial-and-temporal-misalignment-origins-of-integration-challenges}

Spatial and temporal misalignment is a problem that researchers run into
when integrating data from human impact studies with physical exposure
data. For example, physical exposure data on windspeed may have a very
fine resolution, possibly down to seconds or minutes, while data on
birth outcomes may be at a temporal scale of weeks or even months.

Some of the reasons for this are practical. Often physical exposure data
is recorded at monitoring systems that are designed to automatically
record a data point at a fixed interval of time. Human impact data on
health or socioeconomic status comes from administrative sources that
collect data on certain scales. Different sources of data naturally
beget differences in data resolution. Another factor that drives spatial
and temporal misalignment between physical exposure data and human
impacts data is privacy. Physical exposure data is impersonal and can
often represent a very specific temporal and spatial point, whereas
human impacts data is often aggregated to preserve the anonymity and
privacy of study subjects.

The study question that researchers ask may also drive the choice of
spatial and temporal scales used to understand human impacts from
tropical cyclones. If a study is concerned with birth outcomes for
example, having weather data on the windspeed every several seconds may
not be relevant, because birth outcomes related to storm exposure in
utero may operate on a longer time scale. Sometimes exposure to a
tropical cyclone is defined as being present in a county that was hit by
the storm center during a gestational period of a mother's pregnancy as
is the case in (S. C. Grabich et al. 2016). Here, the researchers
divided a pregnancy into exposed and unexposed time to tropical cyclones
after 20 weeks of gestation. Using cumulative storm data therefore gives
the researchers an easier way to integrate physical exposure data with
the gestational period being studied.

Another factor that drives spatial and temporal misalignment in tropical
cyclone studies is that researchers studying human impacts often do not
collect the physical exposure data themselves. Instead they get this
data from large sources of secondary data such as the National Hurricane
Center Data Archive from NOAA (National Oceanic and Atmospheric
Administration) and the NWS (National Weather Service). Researchers also
will use such secondary datasets and sources to compare with primary
data sources. For example in (Lieberman-Cribbin et al. 2017), self
reported flooding exposure data was compared to FEMA flooding exposure
data.

A review of the literature on human impacts of tropical cyclone shows
that certain spatial and temporal scales show up more frequently than
others. Spatially, county level and zip code level data are frequently
used to aggregate health outcomes and other human impacts. Temporal
scales for human impact studies are often cumulative measures of time.
This is in contrast to physical exposure data which is often spatially
at the point location level, and temporally at a very fine resolution
down to minutes and seconds.

In this paper, we will illustrate some some of the situations where
various spatial and temporal levels of physical exposure data are used,
using examples from the literature. These examples will demonstrate
different ways that researchers have integrated physical exposure data
with human impacts data. We will discuss how differences in resolution
between physical exposure and human impact datasets can create
challenges in measuring and inferring the association between tropical
cyclone exposure and human impacts. Finally, we will explore some of the
implications of integrating these different types of data.

\section{Spatial Scales}\label{spatial-scales}

The spatial scale that a researcher uses varies depending on the data
available or sampling method used. In human impacts data finer spatial
scales will correspond more often to individuals or households, while
larger spatial scales will correspond to regions, states, or even
countries. Physical exposure data is often at a small point location or
a grid, based on where weather monitoring sensors are placed. In the
following section we will outline the most common spatial scales used in
tropical cyclone studies and include some examples from the literature
where they were employed.

\subsubsection{Point Location}\label{point-location}

{[}BA: Let's think some about the order we want for these sections.
We're making several good points / analysis here. First, we're defining
what we mean by the resolution (``point location'' here). We probably
want to start with that. Then we have some examples for studies that
have had outcome data at this resolution. Maybe that could go next, to
help illustrate the definition we've given. We've got some information
on \emph{how} the data at this scale was collected (e.g., geocoding from
addresses reported from the study subjects), which I think is really
interesting. Finally, we're got some text that talks about how data at
this resolution could be integrated with some main formats of exposure
data. We might want to end with that (or maybe even, as we work on this
draft, that might go into a different section of the paper).{]}

Point locations are the smallest resolution of spatial data used to
assess the exposure to tropical storms and hurricanes, as they represent
the specific location of individual, non-aggregated observations on the
outcome of interest. In many cases, researchers collect information on
the study subject's residential address through some sort of a survey to
assess point location (Lieberman-Cribbin et al. 2017), (Jaycox et al.
2010),(Bayleyegn et al. 2006). These surveys are often designed to
assess psychological needs of hurricane survivors, as well as medical,
financial, and nutritional needs. For example in (Lieberman-Cribbin et
al. 2017), New York City residents provided their address in a self
reported manner to look at associations between mental health outcomes
and flooding data. This residential address served as a point location
that could be mapped and was compared to flooding data maps created by
FEMA. In other cases, a GPS device is used to record coordinates that
mark a specific point location. An example is (Hagy, Lehrter, and
Murrell 2006), where specific point locations were used to take water
samples were taken to measure parameters of water quality such as
salinity, temperature, dissolved oxygen, and turbidity compared before
and after Hurricane Ivan in Pensacola Bay, Florida. This is a common
practice in ecological research because point locations distributed
across a landscape can be used to observe patterns taking geography into
account. Point locations are also advantageous when using satellite
images in conjuction with analysis of hurricane impact as illustrated in
(Bianchette et al. 2009), where Landsat 5 images were used to compare
vegetation damage, by looking at specific trees at different elevations
to assess the ecological impact of Hurricane Ivan.

The obvious advantage of a point location is that when mapped, it can be
overlayed with physical exposure data on a storm or storms to gage a
very accurate picture of exposure, taking full advantage of high
resolution in the exposure data. Since storm tracks are often spatially
represented by the path of the storm's center, having point locations
for the exposed units of interest allows researchers to more accurately
measure how close each observation was to the storm's central track, and
make further conclusions on this. Similarly, point locations can be
integrated in a straightforward way with gridded exposure data, as might
result from re-analysis datasets or \ldots{} {[}check with James Done
about this{]}, as each point location can be assigned the exposure level
of the closest gridded measurement.

{[}Once we give examples, we should talk about what level the physical
exposure data was recorded as. Did it line up exactly? Grided data. Some
studies avoid the problem by creating a proxy (ex: dist from the storm
track).{]}

\subsubsection{Zip Code/County/Parish}\label{zip-codecountyparish}

While point locations are very useful, many of the papers cited used
larger geographic areas to denote spatial exposure to storms. Zip codes
(Bevilacqua et al. 2020),(Lane et al. 2013), are often used to aggregate
groups of people living in a given area. Counties are at a higher
aggregation level than zip codes (Kinney et al. 2008), (S. C. Grabich et
al. 2016), (S. Grabich et al. 2016), (Schwartz et al. 2018), (Harville
et al. 2010). Often these levels seem to be used when a specific
metropolitan area is being looked at, such as New York City after
Hurricane Sandy (Lane et al. 2013), and Houstan after Hurricane Harvey
(Schwartz et al. 2018).

Aggregating exposure at the county level is convenient because it
utilizes some of the most established methods for assigning exposure
status: the storm track trajectory, and FEMA presidential disaster
declarations (S. Grabich et al. 2016). The storm track trajectory is
typically the path that the tropical cyclone takes, and although the
counties immediately crossed can be categorized as exposed, there are
methods to calculate distance from the storm center that allow for
estimation of exposure at various distances by establishing exposure
thresholds.

There are several disadvantages and pitfalls to using this spatial
level. For one, not all counties and zip codes(which are called parishes
in Louisiana) are the same size or have the same population, so they may
not be immediately comparable. Using the county/parish or zip code makes
it easier for researchers to misclassify exposure. There are many ways
that this can occur in a study on tropical storms; one common example is
that counties selected as exposed are those that had the center of the
storm pass through their county's physical boundaries. However it is
very possible that some individuals lived in a county classified as
exposed based on this criteria, but were in a region of the county far
enough away from the storm center that they were not severely impacted.
These individuals would be classified as exposed when they really were
not and it could bias an apparetn association towards the null.
Alternatively, individuals who lived in a unexposed county, but were
near the border of an exposed county could be incorrectly categorized as
being unexposed even if they actually experienced many of the effects of
the storm.

\subsubsection{State/Metropolitan
Region}\label{statemetropolitan-region}

Many studies used the spatial level of entire states or specific
metropolitan areas to gather information on those who were exposed.
(Harville et al. 2010) is an interesting paper because it looks at the
state level as well as the regional and parish level. In this paper
researchers observed birth outcomes in response to Hurricane Katrina in
the state of Louisiana as a whole, the New Orleans metropolitan area,
and Orleans parish, which is the heart of New Orleans. Looking at these
three levels is a way to compare different incident rates and other
measures of associations across different spatial scales.

The state or national level is the spatial level of an ecological study
and can be useful to compare the emergency preparedness and policies of
different states. The potential for the ecological bias is of course
present when looking at this spatial scale however, which occurs when
the outcomes on the population level (typically an average), do not
represent the individual outcomes very well.

\section{Temporal Scales}\label{temporal-scales}

Thanks to scientitific institutions such as NOAA and the National
Weather Service, there are wide networks of sensors and monitoring
equipment established across the United States that are capable of
recording physical exposure data at a fine enough level as to render it
almost continuous. It is possible to know the wind speed, amount of
rainfall, and air temperature at very fine temporal scales throughout
the duration of a tropical cyclone event. Human impacts data however, is
typically not available at such a fine scale, nor is such a scale
sometimes even relevant. Whereas physical exposure data may be collected
in real time during the storm, many of the human impacts that
researchers are interested in may be only known after the storm, and
thus estimations may have to be made of what happened in the past. Below
are some examples of time scales that are more applicable to a human
scale, and why aggregating physical exposure data to these units of time
may be necessary.

\subsubsection{Day}\label{day}

In the event of a tropical cyclone, there are several situations in
which the temporal unit of a day may be used to analyze exposure.
Physical exposure data from tropical cyclones will typically be
available at this scale anyways, but some studies will look at time
series and use daily exposure data from hospitalizations and visits to
the emergency room.

\subsubsection{Week}\label{week}

Week is a very common unit of time used to ascertain exposure,
particularly for studies that are concerned with birth outcomes and
gestation during hurricane exposure (Kinney et al. 2008), (S. C. Grabich
et al. 2016), (S. Grabich et al. 2016). When the week of gestation is
known, the timing that the hurricane makes landfall, or has its storm
center pass through a county can be matched up to this week of gestation
to identify possible ``critical periods'' of exposure during
development.

In the North Atlantic Basin, tropical cyclones typically occur between
June and November, and so sometimes month to several month periods are
used to aggregate the temporal exposure to them.

\subsubsection{Cumulative Measures of
Time}\label{cumulative-measures-of-time}

It is often the case that specific physical exposures are not considered
in real time to ascertain human impacts. Instead, human impacts are
assessed after the storm has passed, often noting the number of days or
weeks that have passed since the hurricane made landfall. This method is
common when assessing damages, recovery efforts, and when human impacts
are self reported. When this is the case, it is often useful to take an
aggregate exposure an aggregate measure of physical exposure
corresponding to this time frame, such as the maximum wind speed or
maximum flooding level over the period of time being studied.

\section{Implications of not improving this
integration}\label{implications-of-not-improving-this-integration}

When researchers study the economic, social, and health impacts that
tropical storms and hurricanes have in locales like the Gulf Coast of
the United States, it is important to select appropriate spatial and
temporal scales to adequately classify exposure. Misalignment in the
spatial and temporal scale of exposure data versus outcome data creates
challenges when measuring and inferring associations between tropical
cyclone exposures and human impacts. This is a problem because it gives
an inaccurate picture of how communities and individuals' health are
impacted by these storms. Two potential implications are that this
mismatch can introduce bias in estimated associations and that it can
reduce precision in estimates of those associations. In tropical cyclone
studies these biases and reduced precision are usually the result of
misclassification or measurement error, categorizing continuous data,
and ecological bias.

When researchers are confronted with datasets that are at different
temporal or spatial scales, there are typically three options of how to
integrate these datasets. The first option is to aggregate the dataset
that is at a finer resolution in order to match it to the other dataset.
The second option is to interpolate the dataset that is at a more
aggregated level in order to integrate it with a dataset that is at a
finer level. The final option occurs on spatial scales where you have
point levels for all of your datsets but they are spatially misaligned.
In this case, you have to utilize methods of matching those point
locations, for example residential addresses with the closest weather
monitor that measured windspeed or flooding. We will go through each of
these methods and point out the ways in which error, bias, and reduced
precision can occur.

\emph{Aggregating Up}

When researchers have physical exposure data at a very fine resolution,
perhaps even continuous, and human impacts data at a more aggregate
level, it is common and practical to aggregate the physical exposure
data. This same practice could also work the other way around,
aggregating human impacts data to a physical exposure dataset with a
narrower spatial and temporal resolution. In any case, when dealing with
aggregated data of any kind, it is important to realize that information
on the individual level is lost. Researchers should be mindful when
aggregating data, particularly continuous data, that they are losing
information.

Data that are aggregated across a spatial area---for example, the total
number of deaths in a geographic area in a certain time period---is
known as ecological data, aggregate data, or contextual-level data.
Studies that use such data are known as ecological studies (Sedgwick
2014). Ecological bias occurs whenever the aggregate association between
an exposure and an outcome does not properly reflect the association on
the individual level (Greenland and Morgenstern 1989). There are times
when this not a concern, as the research question may be interested in
the population average impact. However, when estimates derived from
ecological studies are used to infer individual estimates, ecological
bias will likely be present. For example, a certain county may be
assigned a specific maximum wind speed, however this is an average and
does not reflect the heterogeneity of maximum wind speeds experienced by
the people living in different parts of the county.

When aggregating data, another concern that arises is misclassification
or measurerement error. Misclassification error occurs when exposure and
outcome variables are measured in categories and the wrong category is
assigned to a particular case/observation - for example when a case that
is exposed is incorrectly categorized as unexposed. Failure to classify
exposure accurately(for example, classifying certain observations as
exposed to a storm when they really were not, or vice-versa), allows
misclassification bias to move the results of the study further from the
true parameter . Measurement error occurs when the variables being
measured are continuous, such as the amount of precipitation or the wind
speed that was measured during a tropical cyclone.

Environmental epidemiology studies are often prone to misclassification
error because the methods of assessing exposure are not always congruent
with the way that researchers conduct human impact studies. It is easy
to map the path of a tropical cyclone's center, and categorize every
county it passes through as an exposed county. However, this information
by itself would not give the researcher any information about population
centers that the storm passed through or near to. A town within an
exposed county may or may not have been close to the storm's path.
Conversely,a town in an unexposed county could be located very close to
the border of an exposed county, and even be closer to the storm's track
than a different town within that exposed county. An example of a study
that could be prone to this kind of bias is (Kinney et al. 2008) where
Louisiana parishes were considered vulnerable to hurricane exposure
based on whether or not the storm center passed through that parish. It
is possible that the cases considered exposed based on living in these
parishes were not in fact exposed since the storm may have passed
through only a certain part of the parish. Never the less, all cases in
a parish are considered exposed or unexposed in the aggregate.

\emph{Interpolating Data}

\emph{Handling Misaligned Point Sources} Sometimes, researchers may have
access to data that is down to the point source, both for physical
exposures and also for human impacts. Very likely however, these point
sources will not be the same. Here the issue is not of integrating
different resolution levels, but rather of matching different point
locations. Weather monitoring sensors may be set up regularly in a grid
formation in a geographic region, but the human impacts point locations
could be tied to a residential address. In this case, the residential
addresses would have to be united with the closest weather monitoring
sensor.

Misclassification is again another potential source of bias in this
situation. A current weather monitor or sensor may give a certain
reading for a maximum wind speed, but it will be the closest weather
monitor for multiple different residential addresses that all experience
different maximum wind speeds.

\emph{Misclassification error / measurement error.} One pathway for
problems is through misclassification / measurement error bias.
Misclassification error occurs when exposure and outcome variables are
measured in categories and the wrong category is assigned to a
particular case/observation - for example when a case that is exposed is
incorrectly categorized as unexposed. Failure to classify exposure
accurately(for example, classifying certain observations as exposed to a
storm when they really were not, or vice-versa), allows
misclassification bias to move the results of the study further from the
true parameter . Measurement error occurs when the variables being
measured are continuous, such as the amount of precipitation or the wind
speed that was measured during a tropical cyclone.

Environmental epidemiology studies are often prone to misclassification
error because the methods of assessing exposure are not always congruent
with the way that researchers conduct human impact studies. It is easy
to map the path of a tropical cyclone's center, and categorize every
county it passes through as an exposed county. However, this information
by itself would not give the researcher any information about population
centers that the storm passed through or near to. A town within an
exposed county may or may not have been close to the storm's path.
Conversely,a town in an unexposed county could be located very close to
the border of an exposed county, and even be closer to the storm's track
than a different town within that exposed county. Where physical
exposure data is often collected at point locations, human impact data
is often at the level of zip code, county, metropolitan area, or state.
It is easy to here how spatially, physical exposure data and human
impact data are collected at different resolutions that increase the
risk for misclassification error.

Another source of misclassification error in tropical cyclone impact
studies is self reported data. Self reported data is used to assess
human impacts that are often not known or apparent until after the
tropical cyclone event. A great example of this is in (Lieberman-Cribbin
et al. 2017) where study subjects were asked to report their own
flooding exposure and their mental health symptoms of depression,
anxiety, and PTSD. It is reasonable to believe that self perceived
exposure to hurricane related flooding would not be independent from
perceived negative mental health symptoms and thus potentially
contribute to differential misclassification error in this situation.

\emph{Dichotomizing continuous exposure measurements.} Sometimes,
researchers use an agreed upon threshold to split a continuous metric
into a binary classification (exposed or unexposed). For example, a
county may be classified as exposed or unexposed based on local winds
exceeding a threshold (e.g.~gale-force winds or higher). S.C. Grabich et
al. 2016 classified hurricane exposure in a Florida county using maximum
wind speed. Maximum wind speed is a continuous variable, but the study
used binary categorizations to divide it into tropical wind speeds,
classified as greater than 39 miles per hour, and hurricane wind speeds,
classified as greater than 74 miles per hour. Florida counties
experiencing maximum wind speeds below 39 miles per hour were considered
unexposed.

Researchers typically dichotomize or categorize continuous variables in
several situations for several reasons. They do this typically because
it simplifies the data and allows for easier analysis and interpretation
(Naggara et al. 2011). Additionally, it is very common in clinical
settings to categorize continuous variables, for example hypertensive or
not hypertensive, overweight or not overweight, dead or alive, etc. (Van
Walraven and Hart 2008).

Despite several advantages to dichotomizing continuous variables that we
just discussed, the general consensus in epidemiology is not to do it.
Statistical power is lost because so much information is lost when
categorization occurs (Van Walraven and Hart 2008). This makes sense
when you consider that continuous variables allow you to observe nuance
in the data and perceive a dose response relationship between the
predictor and response variables, should one exist. This effect is
masked when researchers categorize data, and even more so when a smaller
number of categorical variables are used (for example dichotomization
itself at 2). Generally, if you are going to categorize continuous data,
it is better to use 3 or more categories rather than just two, because
this will capture more of what the data that would otherwise be lost. An
example of a paper that used three different bins was (Kinney et al.
2008), which explored the risk of autism after a pregnancy that included
exposure to a tropical storm in the state of Louisiana. The study
authors classified tropical storm exposure as severe, intermediate, and
low exposure, and these exposure classifications were determined based
on whether a mother lived in a Louisiana parish that had both of the
exposure factors of interest: storm intensity and storm vulnerability.
Storm vulnerability in this case was based on another dichotomy: whether
or not the storm center passed through the parish of interest. Storm
vulnerability was a measure of how vulnerable the inhabitants of the
parish were to the effects of a storm (higher socioeconomic
neighborhoods and parishes have more resources to withstand and recover
from a tropical storm for example).

Another obvious problem with categorizing continuous data is that the
cutoff points are often arbitrary. In the case of dichotomization, the
median is often used, but there is typically no reason to assume that
the median is a reasonable cutoff point. Because different samples will
have different medians, this automatically makes many categorical bins
difficult to compare across studies (Altman and Royston 2006). Further,
choosing optimal cutoff points that give the smallest p-values can lead
to spurious results (Altman and Royston 2006).

Not surprisingly, dichotomizing continuous variables can bias results. A
study by Selvin showed that the odds ratios can be significantly
different depending on the chosen cutoff that is implemented in a study
(Van Walraven and Hart 2008). Categorical variables can also put
otherwise similar observations into separate bins if they are close but
on opposite sides of the cutoff (Altman and Royston 2006). Choosing a
median as a cutoff is intended to delineate bins, but if the bins are a
``high'' and ``low'' group, two individual observations that may only be
a fraction different, but on either sides of the mean, will be
classified as high and low respectively, and give the false impression
that they are significantly different.

While dichotomizing continuous variables is something that can be done
for either the exposure or the outcome of interest in a study, for our
purposes we are primarily interested in continuous \emph{exposures}.
This means that we are primarily interested in the effects of
dichotomizing variables such as wind speed, rainfall, temperature,
distance from storm center, and distance from coastline, among other
factors. Many epidemiology studies will dichotomize continuous outcome
variables such as blood pressure, body weight (BMI), and length of
pregnancy in order to gage medical concern and priorities, but because
we are concerend with creating a data framework that makes storm
exposure data accessible for epidemiologists, exposure scientits,
economists, and other scientists to use, we have a priority to look at
exposure variables.

\emph{Scales for Categorizing Wind Speeds} There are several methods in
existence for categorizing wind speed, one of the most frequently used
variables for estimating exposure to hurricanes and tropical storms. The
first is the Saffir-Simpson scale, which uses five different bins to
classify varying levels of wind speed and determine the severity of a
storm. The first level, Category 1 is designated for hurricanes and
tropical storms with maximum wind speeds of between 64 - 82 knots and is
generally considered dangerous to people, livestock, and pets from the
hazard of flying and falling debris (Taylor et al. 2010). On the higher
end of the scale, Category 5 designates hurricanes with maximum wind
speeds above 137 knots and is considered to have catastrophic effect on
damage and a high probability of injury or death to people, livestock,
and pets even if they are sheltering indoors (Taylor et al. 2010).

Forecasters classify hurricanes into categories on the Saffir-Simpson
scale based on maximum sustained surface wind speed. This is defined as
the peak one minute wind speed at a height of 10 feet over an
unobstructed exposure (Taylor et al. 2010). An important limitation of
the Saffir-Simpson scale is that it doesn't account for other
hurricane-related impact variables such as storm surges, flooding, and
tornadoes (Taylor et al. 2010).

Another scale used to categorize wind speed is the Beaufort scale,
created by Admiral Sir Francis Beaufort, used to classify wind speeds
both over land and sea. While the Saffir-Simpson scale is only
designated for wind speeds that are already at hurricane levels (greater
than 64 knots), the Beaufort scale considers the wind speeds below this.
The scale ranges from Force 0 (0-1 knots and calm) to Force 12 (64 to 71
knots and hurricane). Other interesting parts of the scale include Force
3 (4-6 knots) which is a gentle breeze, and Force 8 (34-40 knots) which
is considered a gale.

Categorizing wind speeds presents researchers with some of the same
problems mentioned above that happen when dealing with continuous data,
but both scales are based off associations between winds at certain
speeds and observed damage and health impacts to communities exposed to
these wind speeds.

\emph{Aggregate Hurricane Exposure Metrics}

Another method of assessing damage and impact of tropical storms and
hurricanes is through a single aggegrate exposure metric. While
aggregate values often represent the mean of all the values recorded,
weather data is typically assessed by the maximum value. This could be
something like the maximum wind speed reached in a particular county or
parish, or the total monetary cost in damage due to flooding in a
metropolitan statistical area. The Saffir-Simpson scale is an example of
how entire storms are often classified by their maximum wind speed.

Although using a single exposure value can simplify analysis and
interpretation, particularly over an extended temporal scale, there are
some obvious drawbacks to relying on one single aggregate value. For
example, the Saffir Simpson categories typically correspond only to the
geographic point location where the maximum wind speed was observed
(Taylor et al. 2010). Hurricane Wilma in 2005 for example, was a
Category 3 hurricane when it made landfall on the southwest coast of
Florida, but it created Category 1 and Category 2 conditions for the
more populous Miami-Dade, Broward, and Palm Beach counties when it
finally reached them (Taylor et al. 2010).

Single exposure metrics are often used after a storm event has happened.
They are very common in assessing ecological damage after a large
hurricane.

\emph{Ecological Bias/The Ecological Fallacy} Because studying tropical
storm and hurricane exposures requires us to look at different spatial
scales, we run the risk of encountering the ecological bias when looking
at larger spatial aggregations. Ecological bias occurs whenever the
aggregate association between an exposure and an outcome does not
properly reflect the association on the individual level (Greenland and
Morgenstern 1989). Ecological studies themselves don't look at
individuals, but rather at an aggregate value, usually within a defined
geographic region. Looking at national levels of obesity, cancer, or
life expectancy, and comparing countries with respect to these outcomes
and some exposure is an example of what ecological studies aim to
achieve.

An example of a study that could be prone to this kind of bias is
(Kinney et al. 2008) where Louisiana parishes were considered vulnerable
to hurricane exposure based on whether or not the storm center passed
through that parish. It is possible that the cases considered exposed
based on living in these parishes were not in fact exposed since the
storm may have passed through only a certain part of the parish. Never
the less, all cases in a parish are considered exposed or unexposed in
the aggregate.

\begin{center}\rule{0.5\linewidth}{0.5pt}\end{center}

{[}BA: I'm adding some additional text/notes we can work into Claim 2 as
appropriate. I drafted these while working on another manuscript but
they were more detailed than we needed there, so we can work them in
here.{]}

Measurement error can be either random or systematic. Systematic error
can often be corrected with adjustment if the direction and typical size
of the error is understood. Random error cannot in the same way. Either
type of measurement error can be either differential (associated with
the probability of the outcome) or non-differential (independent of the
distribution / probability of the outcome). In simpler models, this
characteristic might help in predicting whether the resulting bias is
likely toward the null; however, more complex models (e.g., statistical
models with adjustment for potential confounders) are trickier to
diagnose in terms of the likely implications of differential versus
non-differential measurement error {[}would need a ref for this{]}.

When a single value of exposure is assigned across an aggregated level
(e.g., a single exposure measurement for a county or ZIP code), it
assumes constant exposure across that area. However, this will typically
not be the case---hazards like storm-associated wind, rain, and flooding
can vary in intensity across these spatial areas. For some hazards, this
variation can be notable. Storm surge, for example, will typically be
limited to coastal areas of a county or ZIP code. Other hazards, like
storm-associated wind and rain, are more likely to be more homogeneous
across space, and so have less within-county/ZIP code variation. The
rainfields for tropical cyclones are very large, and while there are
rainbands within the storm that might have particularly high rates of
precipitation, these progress over the course of the storm, and it is
unlikely that a county will have one area that experienced very extreme
precipitation while another experienced very little {[}BA: We could see
if we could find a good ref. or two on this point{]}. Similarly, while
topographic features and other variability can create variation in the
sustained and gust windspeeds experienced in an area from a storm, it is
unlikely that one part of a county would experience high-impact winds
from a storm while other parts of the county experienced mild wind
{[}BA: We could look for a ref for this, too{]}.

If you use a single exposure estimate for everyone in an area, there is
the chance that some people within that area will be misclassified (if
exposure is measured as exposed/unexposed) or have exposure measured
with error (if a continuous metric of exposure is being used), unless
the exposure is perfectly homogeneous across the area. This exposure
misclassification or measurement error can lower the power of the study
to detect a clear association between exposure to a storm hazard and a
certain societal impact, as this smoothing drops information inherent in
the within-county variation in exposure levels. It can also bias
estimates of the association between exposure and outcome in the same
way exposure misclassification through any other mechanism would.

When a proxy exposure estimate (e.g., county-level average exposure
level) is used for a group of individuals in the study, it can result in
a type of exposure measurement error called Berkson error. In this case,
the true exposure of each individual is randomly distributed around the
proxy or mean exposure level assigned to him or her. In other words, the
group as a whole is assigned a common exposure level, based on the
average exposure across that group, when in fact the individuals' true
exposure levels are randomly distributed around this common assigned
exposure level. {[}BA: I think this type of error might be a risk when
aggregating exposure data, but we should look into it a bit more to make
sure I'm right.{]}

It can be important to think about the scale at which the process
happens. For something very local (e.g., aggregating to a very small
neighborhood scale), much less information will be lost compared to
aggregated to a large scale (e.g., state). If an exposure tends to be
fairly homogenous across the spatial scale used for aggregation, then
these concerns are lessened (Wakefield and Haneuse 2008).

\begin{center}\rule{0.5\linewidth}{0.5pt}\end{center}

Data that are aggregated across a spatial area---for example, the total
number of deaths in a geographic area in a certain time period---is
known as ecological data, aggregate data, or contextual-level data.
Studies that use such data are known as ecological studies (Sedgwick
2014).

Aggregated or ecological data can be used to infer a contextual effect,
for example. Sometimes, however, aggregated data are used to infer
individual-level associations. While the first type of inference seeks
to answer questions like how the county-wide rate of an outcome of
interest changes when the county is exposed to a storm hazard, the
second seeks to determine how a person's individual risk of an outcome
changes if he or she is personally exposed to the hazard. The second
type of inference can be prone to bias that results from cross-level
inference---the data used to model the association is at the contextual
level (e.g., county-level) while the inference is for the individual
association between exposure and outcome. When an individual-level
association is estimated from ecological data, the estimate can be very
biased from the true association, event to the point of reversing the
effect estimate---estimating a protective effect, for example, when the
true effect is detrimental (Wakefield and Haneuse 2008). This type of
bias is called ecological or cross-level bias (Greenland and Robins
1994; Idrovo 2011), and the misconception that associations estimated
from data at the ecological/aggregated level provide an unbiased
estimate of individual-level associations between exposure and risk of
the outcome is called the ecological fallacy (Wakefield and Shaddick
2006; Portnov, Dubnov, and Barchana 2007).

Ecological bias can result both from individual-level exposure
measurement error inherent in assigning a common exposure estimate to
everyone in an area, while the exposure varies in intensity across that
area. It can also result from confounding, even if the confounders are
controlled at the ecological level. When data are aggregated across a
spatial area, information is lost about how all relevant
factors---exposure level, outcome risk, confounders, and even potential
effect modifiers---vary within that spatial area. Just as aggregation
smooths over within-area variation in exposure levels, it also smooths
over within-area variation in levels of potential confounders. Depending
on the patterns of this within-area variation, a result could be that
ecological-level control of the confounders does not, in fact, control
for their role at the individual level, and so the association inferred
at the ecologic level continues to be confounded by them when inferred
to the individual level. In other words, a factor could still confound
the inference of an individual-level association, even if it is
controlled at a population level in an ecological model. For example, a
study of the association between risk of pre-term birth and tropical
cyclone exposure could control for county-level smoking when modeling
county-level storm exposure and county-level rates of pre-term births.
Even with this control, an observed association could result from
differences in individual smoking status, if there is within-county
variation in smoking and if this has a different pattern across people
in the county than variation in exposure from the county-wide exposure
estimate.

If individual-level inference is the aim, and population-level data is
available, there are some methods for using it while still aiming to
avoid ecological bias. Indeed, it can be helpful to use population-level
data, as it is often available for a large population, improving the
power and precision of the study (Wakefield and Haneuse 2008; Wakefield
and Shaddick 2006). Further, the level of exposure might vary a lot more
over the population captured with population-level data compared to the
variation that captured in a smaller sample of individual-level data
(Wakefield and Haneuse 2008). This can contribute both to statistical
power and improve external validity (as the study data will cover more
of the range of exposure that might ever be expected). There are ways,
for example, to supplement population-level data with samples of
individual-level data through two-level, semi-ecologic study designs
(Wakefield and Haneuse 2008). Other study designs can also be used to
leverage ecological data while minimizing risk from ecological bias. For
example, potential confounders like age distribution and smoking rates
vary much less within a county over time than comparing between
counties. Time series-style study designs, which compare a county to
itself over time, therefore allow for very similar covariate
distributions between exposure and non-exposure. This can help since the
mechanism for ecological bias depends on the joint distribution between
individual exposure, outcome, and covariates, if any are included in the
model. Other studies add to this design by stabilizing for temporal
confounding through the addition of counties that were never exposed,
allowing for a differences-in-differences style approach to calibrate
for seasonal or longer-term trends that might otherwise create
confounding. For example, many health outcomes have a strong seasonal
trend, with peak rates in the winter and lows in the summer. Since the
hurricane season stretches from summer into fall, a study design that
compares the rate of a health outcome in an exposed county to the rate
two weeks before the exposure might be biased away from the null, since
baseline rates of the health outcome will typically be moving up over
most of the hurricane season.

Ecological bias can also complicate estimation of effect modification,
which otherwise could help in identifying vulnerabilities and
susceptibilities among certain subpopulations (Wakefield and Haneuse
2008).

For disasters, there are added nuances. First, in some cases, the
ecologic-level effect (contextual effect) will be directly of interest.
For example, public health planners in a city may be more interested in
knowing how a storm hazard exposure is likely to change city-wide rates
of certain outcomes than in how it would change individual-level risk.
In this case, it is appropriate to use of ecologic-level data, and
resulting estimates will not be prone to ecological bias) (Idrovo 2011;
Greenland and Robins 1994), although when inferring contextual effects
from ecologic data without considering individual-level factors, there
is a chance for the \emph{sociologistic fallacy} (Idrovo 2011). (BA: In
this last sentence, the cited article is a letter to the editor, and it
in turn is referencing some other papers that more fully define these
ideas. If we keep this, we should go back and read more deeply into
those cited papers from the Idrovo 2011 reference).

Second, for a disaster, the relevant exposure might be not just at the
individual level (e.g., winds or flooding at the individual's
residence), but also throughout a broader area surrounding the
individual. Disasters bring physical hazards that can harm people
directly, but also through indirect pathways. The causal pathways for
tropical cyclones to affect human health and cause other societal
impacts therefore differ from those for a dangerous substance, like air
pollutants, in which the substance itself must enter the body to cause
harm. While some health risk comes directly from the storm (e.g., deaths
and injuries from trees falling on homes or drowning from flooding),
there are many more pathways that are indirect. These include pathways
that go through the way that the storm's damage affects community
infrastructure and access to medical care. For example, a tropical
cyclone can bring high winds that cause power outages, and as a result
those affected could be exposed to more outdoor hazards (outdoor air
pollution, heat), struggle to safely store perishable food and
medications, and lose means to power medical equipment. While extreme
winds at a person's residence would increase their risk of a power
outage, outages could also be caused by damage to the grid in another
part of the community. In some cases, then, the level of exposure in a
person's community may be as important in opening a pathway of risk as
exposure at the person's immediate location.

Finally, if the disaster has a large health impact, the health outcome
of one person in the community could affect the risk of the outcome (or
other adverse outcomes) for others. This situation is often only the
case for infectious diseases, where one person with the disease can
spread it to others. However, if the community-wide impact is large
enough, it can affect access to and effectiveness of medical care for
everyone in the community. if hospitals in the community are over
capacity or have to evacuate, this could increase health risk for people
in a fairly large ``catchment'' area for that hospital. This effect has
been seen recently with Covid 19---attempts to ``flatten the curve'' aim
to avoid moving into a state where a community's health system becomes
overwhelmed and can no longer deliver a typical level of care to those
in the community. This effect could happen with either infectious or
non-infectious diseases. Also, there may be confounders that are
relevant at the contextual, rather than individual level, as well as
modifiers. For example, whether the county is coastal could be a
contextual-level confounder and effect modifier. This will influence
whether the county is exposed to that storm or not, since storms usually
weaken rapidly when the center is over land. In terms of confounding
pathways, coastal communities might tend to have lower levels of air
pollution, because sea breezes clear the pollution regularly. They might
also be wealthier on average, since property on or near the beach is
desirable. Finally, they might be better prepared for or more hardened
against tropical cyclones at the community-wide level (e.g., through
hardier power infrastructure, more rigorous building codes, higher
likelihood of evacuating in advance of a threatening storm) compared to
nearby inland counties.

When inferring individual-level associations from individual-level data,
without considering an additional role of ecological-level factors, this
is known as the \emph{psychologistic} or \emph{individualistic fallacy}
(Idrovo 2011). (BA: In this last sentence, the cited article is a letter
to the editor, and it in turn is referencing some other papers that more
fully define these ideas. If we keep this, we should go back and read
more deeply into those cited papers from the Idrovo 2011 reference).

\begin{center}\rule{0.5\linewidth}{0.5pt}\end{center}

\section{Discussion}\label{discussion}

???

\section{Terms}\label{terms}

These are terms we're using right now that we might want to iterate on,
in conjunction with our colleagues on the project, to make sure we have
terms that are precise and consistent across the document:

\begin{itemize}
\tightlist
\item
  \textbf{physical exposure data}: By this, we mean things that are
  measured about the storm like wind speed, rainfall, measures of
  flooding, and other things that might be considered more in the realm
  of what an atmospheric scientist or engineer might measure about the
  storm. We're contrasting this with data that for human impacts studies
  on outcomes among humans (e.g., pregnancy outcomes, economic outcomes
  like unemployment)
\item
  \textbf{resolution}: We're using this right now to talk about spatial
  and temporal levels of aggregation. Sometimes, we're using ``scales''
  instead, I think.
\item
  **
\end{itemize}

Annual Reviews of Microbiology Annual Reviews of Statistics
Annualreviews.org \textless{} Good for learning about stuff for
interdisciplinary work.

(S. C. Grabich et al. 2016) was another paper that looked at birth
outcomes after tropical storms. The researchers in this case found a
positive association between exposure to a hurricane and the risk of a
pre-term birth.

(Bevilacqua et al. 2020) also found higher levels of PTSD, as well as
probable depression and anxiety among residents with a higher Hurricane
Exposure Score in Houston, Texas. Displaced Puerto Ricans living in
Florida after Hurricane Maria also exhibited higher rates of depression,
anxiety, and PTSD(Scaramutti et al. 2019). These mental health outcomes
were compared to Puerto Ricans living on the island, and the individuals
who had migrated reported higher frequencies of mental health problems
than those who had not. Displacement after a tropical storms is a common
human impact that leads to other mental health effects, as well as
economic, social, and environmental effects.

Tropical cyclones often disrupt local economies of coastal communities.
For example, in Florida, hurricanes lead to demand shocks in the economy
with a positive net effect on earnings and negative net effect on
employment - counties directly hit by hurricanes experienced up to
4.35\% increases in earnings and 4.76\% decreases in employment (Belasen
and Polachek 2008). This confusing paradox makes sense when one
considers that while hurricanes may wipe out local businesses, the
post-hurricane recovery period boosts certain businesses and sectors.
New Orleans in the aftermath of Hurricane Katrina is evidence of this.
While the city initially experienced the negative effects of a shut down
economy, several years after the storm revealed that victims of
Hurricane Katrina in New Orleans experienced increased income relative
to cities not affected by the storm, perhaps due to a strengthed labor
market in post-Katrina New Orleans (Deryugina, Kawano, and Levitt 2018).
High costs of rebuilding infrastructure, providing resources to
displaced populations, medical bills, and loss of businesses after a
tropical storm all drive these economic burdens.

Non-differential misclassification refers to misclassification of either
the exposure or the outcome, that is unrelated to the other (Aschengrau
and Seage 2013). The effect of misclassifying exposures will often,
though not always, bias the results of outcome towards the null
(Armstrong 1998). In effect, this will weaken or obscure any
associations that are present that the researcher may hope to observe in
the data (Armstrong 1998).

Differential misclassification error occurs when the misclassification
of the outcome is related to the misclassification of the exposure or
vice versa (Aschengrau and Seage 2013). While non-differential
misclassification often (though not always) has the effect of moving the
observed association or parameter towards the null, differential
misclassification can move the observation in either direction.

\section*{References}\label{references}
\addcontentsline{toc}{section}{References}

\hypertarget{refs}{}
\hypertarget{ref-altman2006cost}{}
Altman, Douglas G, and Patrick Royston. 2006. ``The Cost of
Dichotomising Continuous Variables.'' \emph{Bmj} 332 (7549). British
Medical Journal Publishing Group: 1080.

\hypertarget{ref-aschengrau2013essentials}{}
Aschengrau, Ann, and George R Seage. 2013. \emph{Essentials of
Epidemiology in Public Health}. Jones \& Bartlett Publishers.

\hypertarget{ref-bayleyegn2006rapid}{}
Bayleyegn, Tesfaye, Amy Wolkin, Kathleen Oberst, Stacy Young, Carlos
Sanchez, Annette Phelps, Joann Schulte, Carol Rubin, and Dahna Batts.
2006. ``Rapid Assessment of the Needs and Health Status in Santa Rosa
and Escambia Counties, Florida, After Hurricane Ivan, September 2004.''
\emph{Disaster Management \& Response} 4 (1). Elsevier: 12--18.

\hypertarget{ref-belasen2008hurricanes}{}
Belasen, Ariel R, and Solomon W Polachek. 2008. ``How Hurricanes Affect
Wages and Employment in Local Labor Markets.'' \emph{American Economic
Review} 98 (2): 49--53.

\hypertarget{ref-bevilacqua2020understanding}{}
Bevilacqua, Kristin, Rehana Rasul, Samantha Schneider, Maria Guzman,
Vishnu Nepal, Deborah Banerjee, Joann Schulte, and Rebecca M Schwartz.
2020. ``Understanding Associations Between Hurricane Harvey Exposure and
Mental Health Symptoms Among Greater Houston-Area Residents.''
\emph{Disaster Medicine and Public Health Preparedness}. Cambridge
University Press, 1--8.

\hypertarget{ref-bianchette2009ecological}{}
Bianchette, TA, K-B Liu, NS-N Lam, and LM Kiage. 2009. ``Ecological
Impacts of Hurricane Ivan on the Gulf Coast of Alabama: A Remote Sensing
Study.'' \emph{Journal of Coastal Research}. JSTOR, 1622--6.

\hypertarget{ref-deryugina2018economic}{}
Deryugina, Tatyana, Laura Kawano, and Steven Levitt. 2018. ``The
Economic Impact of Hurricane Katrina on Its Victims: Evidence from
Individual Tax Returns.'' \emph{American Economic Journal: Applied
Economics} 10 (2): 202--33.

\hypertarget{ref-grabich2016measuring}{}
Grabich, SC, J Horney, C Konrad, and DT Lobdell. 2016. ``Measuring the
Storm: Methods of Quantifying Hurricane Exposure with Pregnancy
Outcomes.'' \emph{Natural Hazards Review} 17 (1). American Society of
Civil Engineers: 06015002.

\hypertarget{ref-grabich2016hurricane}{}
Grabich, Shannon C, Whitney R Robinson, Stephanie M Engel, Charles E
Konrad, David B Richardson, and Jennifer A Horney. 2016. ``Hurricane
Charley Exposure and Hazard of Preterm Delivery, Florida 2004.''
\emph{Maternal and Child Health Journal} 20 (12). Springer: 2474--82.

\hypertarget{ref-greenland1989ecological}{}
Greenland, Sander, and Hal Morgenstern. 1989. ``Ecological Bias,
Confounding, and Effect Modification.'' \emph{International Journal of
Epidemiology} 18 (1). Oxford University Press: 269--74.

\hypertarget{ref-greenland1994invited}{}
Greenland, Sander, and James Robins. 1994. ``Invited Commentary:
Ecologic Studies---biases, Misconceptions, and Counterexamples.''
\emph{American Journal of Epidemiology} 139 (8). Oxford University
Press: 747--60.

\hypertarget{ref-hagy2006effects}{}
Hagy, James D, John C Lehrter, and Michael C Murrell. 2006. ``Effects of
Hurricane Ivan on Water Quality in Pensacola Bay, Florida.''
\emph{Estuaries and Coasts} 29 (6). Springer: 919--25.

\hypertarget{ref-harville2010population}{}
Harville, Emily W, Tri Tran, Xu Xiong, and Pierre Buekens. 2010.
``Population Changes, Racial/Ethnic Disparities, and Birth Outcomes in
Louisiana After Hurricane Katrina.'' \emph{Disaster Medicine and Public
Health Preparedness} 4 (S1). Cambridge University Press: S39--S45.

\hypertarget{ref-idrovo2011three}{}
Idrovo, Alvaro J. 2011. ``Three Criteria for Ecological Fallacy.''
\emph{Environmental Health Perspectives} 119 (8). National Institute of
Environmental Health Sciences: a332--a332.

\hypertarget{ref-jaycox2010children}{}
Jaycox, Lisa H, Judith A Cohen, Anthony P Mannarino, Douglas W Walker,
Audra K Langley, Kate L Gegenheimer, Molly Scott, and Matthias Schonlau.
2010. ``Children's Mental Health Care Following Hurricane Katrina: A
Field Trial of Trauma-Focused Psychotherapies.'' \emph{Journal of
Traumatic Stress: Official Publication of the International Society for
Traumatic Stress Studies} 23 (2). Wiley Online Library: 223--31.

\hypertarget{ref-kinney2008autism}{}
Kinney, Dennis K, Andrea M Miller, David J Crowley, Emerald Huang, and
Erika Gerber. 2008. ``Autism Prevalence Following Prenatal Exposure to
Hurricanes and Tropical Storms in Louisiana.'' \emph{Journal of Autism
and Developmental Disorders} 38 (3). Springer: 481--88.

\hypertarget{ref-lane2013health}{}
Lane, Kathryn, Kizzy Charles-Guzman, Katherine Wheeler, Zaynah Abid,
Nathan Graber, and Thomas Matte. 2013. ``Health Effects of Coastal
Storms and Flooding in Urban Areas: A Review and Vulnerability
Assessment.'' \emph{Journal of Environmental and Public Health} 2013.
Hindawi.

\hypertarget{ref-lieberman2017self}{}
Lieberman-Cribbin, Wil, Bian Liu, Samantha Schneider, Rebecca Schwartz,
and Emanuela Taioli. 2017. ``Self-Reported and Fema Flood Exposure
Assessment After Hurricane Sandy: Association with Mental Health
Outcomes.'' \emph{PLoS One} 12 (1). Public Library of Science: e0170965.

\hypertarget{ref-naggara2011analysis}{}
Naggara, O, J Raymond, F Guilbert, D Roy, A Weill, and Douglas G Altman.
2011. ``Analysis by Categorizing or Dichotomizing Continuous Variables
Is Inadvisable: An Example from the Natural History of Unruptured
Aneurysms.'' \emph{American Journal of Neuroradiology} 32 (3). Am Soc
Neuroradiology: 437--40.

\hypertarget{ref-portnov2007ecological}{}
Portnov, Boris A, Jonathan Dubnov, and Micha Barchana. 2007. ``On
Ecological Fallacy, Assessment Errors Stemming from Misguided Variable
Selection, and the Effect of Aggregation on the Outcome of
Epidemiological Study.'' \emph{Journal of Exposure Science \&
Environmental Epidemiology} 17 (1). Nature Publishing Group: 106--21.

\hypertarget{ref-scaramutti2019mental}{}
Scaramutti, Carolina, Christopher P Salas-Wright, Saskia R Vos, and Seth
J Schwartz. 2019. ``The Mental Health Impact of Hurricane Maria on
Puerto Ricans in Puerto Rico and Florida.'' \emph{Disaster Medicine and
Public Health Preparedness} 13 (1). Cambridge University Press: 24--27.

\hypertarget{ref-schwartz2018preliminary}{}
Schwartz, Rebecca M, Stephanie Tuminello, Samantha M Kerath, Janelle
Rios, Wil Lieberman-Cribbin, and Emanuela Taioli. 2018. ``Preliminary
Assessment of Hurricane Harvey Exposures and Mental Health Impact.''
\emph{International Journal of Environmental Research and Public Health}
15 (5). Multidisciplinary Digital Publishing Institute: 974.

\hypertarget{ref-sedgwick2014ecological}{}
Sedgwick, Philip. 2014. ``Ecological Studies: Advantages and
Disadvantages.'' \emph{Bmj} 348. British Medical Journal Publishing
Group.

\hypertarget{ref-taylor2010saffir}{}
Taylor, Harvey Thurm, Bill Ward, Mark Willis, and Walt Zaleski. 2010.
``The Saffir-Simpson Hurricane Wind Scale.'' \emph{Atmospheric
Administration: Washington, DC, USA}.

\hypertarget{ref-van2008leave}{}
Van Walraven, Carl, and Robert G Hart. 2008. ``Leave'em Alone-Why
Continuous Variables Should Be Analyzed as Such.''
\emph{Neuroepidemiology} 30 (3). S. Karger AG: 138.

\hypertarget{ref-wakefield2008overcoming}{}
Wakefield, Jon, and Sebastien J-PA Haneuse. 2008. ``Overcoming Ecologic
Bias Using the Two-Phase Study Design.'' \emph{American Journal of
Epidemiology} 167 (8). Oxford University Press: 908--16.

\hypertarget{ref-wakefield2006health}{}
Wakefield, Jon, and Gavin Shaddick. 2006. ``Health-Exposure Modeling and
the Ecological Fallacy.'' \emph{Biostatistics} 7 (3). Oxford University
Press: 438--55.

\end{document}
